\chapter{Allgemeines}
\label{chap:allgemeines}

\section{Zweck des Dokuments}
\label{sec:allgemeines.ZweckdesDokuments}
Das Pflichtenheft beschreibt die Ziele, welche mit der angestrebten L�sung der Aufgabe
der Bachelor Thesis zu erreichen sind, sowie Zeitplan und Inhalt der Dokumentation.

\section{Ausgangslage}
\label{sec:allgemeines.Ausgangslage}
Im vorg�ngigen Modul Projekt 2 haben wir Vorarbeiten zur Bachelor Thesis durchgef�hrt. Wir haben uns in das Thema k�nstliche Intelligenz eingearbeitet und uns die AI Challenge Ants genauer angeschaut. Die Pfadsuche wurde vertieft angeschaut. Die Algorithmen A* und HPA* wurden implementiert. Wir haben auf einen modularen und strukturierten Aufbau unseres Bots geachtet, damit die Bachelorarbeit auf dem Code aufbauen kann.

\section{Ziel des Moduls 7321 Bachelor Thesis}
\label{sec:allgemeines.ZieldesModulsBachelorThesis}

Dieses Modul hat zum Ziel, dass die Studierenden
\begin{itemize}
\item selbst�ndig und verantwortungsbewusst in kleinen Gruppen an einem Projekt arbeiten k�nnen;
\item die erworbenen F�higkeiten aus dem Schwerpunktfach bzw. aus dem Studium im Projekt anwenden k�nnen;
\item die grunds�tzlichen Prinzipien des Ablauf eines Software-Projekts in einem gr�sseren konkreten Projekt selbst�ndig einsetzen k�nnen;
\item in der Lage sind, einen umfassenden Bericht zu ihrer Arbeit zu verfassen, die Resultate m�ndlich zu pr�sentieren und ihre Arbeit zu verteidigen.
\end{itemize}
Quelle: http://www.ti.bfh.ch/fileadmin/modules/BTI7321-de.xml

\section{Aufgabenstellung}
\label{sec:allgemeines.Aufgabenstellung}
Jedes Jahr findet der von der Universit�t Waterloo veranstaltete Wettbewerb AI Challenge statt. Ziel des Wettbewerbs, der im Turniermodus durchgef�hrt wird, ist es, ein KI-Programm zu entwickeln, das ein speziell f�r diesen Wettbewerb entworfenes Computerspiel bestm�glich meistert. Im letztj�hrigen Turnier galt es im Computerspiel Ants eine Ameisenkolonie zu steuern.

In dieser Bachelorarbeit soll auf der bestehenden Projektarbeit aufgesetzt werden, in der ein Programm f�r das letztj�hrige Computerspiel Ants entwickelt wurde. Ziel dabei ist es nicht nur ein ganz konkretes Computerspiel zu meistern, sondern auch allgemein verwendbare KI-Methoden, wie beispielsweise die Pfadsuche, im Rahmen eines Frameworks zur Verf�gung zu stellen. Dabei bleibt noch offen, ob das bestehende Computerspiel Ants weiterentwickelt wird oder ob das aktuelle Computerspiel des AI Challenge 2012 realisiert wird. Ein Schwerpunkt dieser Bachelorarbeit k�nnte auf der Verfeinerung von Pfadsuchverfahren und auf der Erprobung und Realisierung von Influence-Maps liegen.

Quelle: https://www2.ti.bfh.ch/fbi/2013/Studienbetrieb/BaThesisHS12/aufgabestellungen/IERJ1-3-12-de.html


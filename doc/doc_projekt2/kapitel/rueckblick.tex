\chapter{R�ckblick}
\label{chap:rueckblick}

\section{Resultate}
\label{sec:rueckblick_resultate}
Wir haben mit dieser Arbeit einen guten Einblick in die Programmierung von Bots erhalten. Uns ist bewusst geworden, dass viele Aspekte zusammenspielen m�ssen, damit sich die einzelnen Ameisen intelligent verhalten. Da die Spielschnittstelle einfach gehalten und im Web gut dokumentiert ist kamen wir relativ schnell zu einem sichtbaren Resultat, indem bereits erste Spiele verfolgt werden konnten. Mit einem sauberen Aufbau des Programmcodes konnten wir die Aufgaben einfach gliedern und haben einen guten Grundstein gelegt, falls wir dieses Projekt als Bachelorarbeit weiterverfolgen. Dank den eingesetzten Pfadsuchalgorithmen konnten wir die Rechenzeit verringern und haben so w�hrend einem Zug genug Zeit um weitere Berechnungen zu machen. 

\section{Herausforderungen}
\label{sec:spielbeschrieb_herausforderungen}
Beim Suchen eines Programmierfehlers musste das Logfile durchforstet werden, was sich als eine aufw�ndige und zeitraubende Arbeit herausstellte. Dadurch sind wir auch auf die Idee des Javascript Addon gekommen. Zu Beginn der Arbeit wurde die zur Verf�gung stehende Zeit eines Zuges rasch aufgebraucht, da wir keinen schlauen Algorithmus f�r die Pfadsuche verwendeten; dies konnte mit A* und HPA* behoben werden.

\section{Ziele f�r Bachelorarbeit}
\label{sec:spielbeschrieb_zieleBachelorarbeit}
Im Projekt 2 haben wir uns vor allem auf die Pfadsuche konzentriert. Die Spiele-Entwicklung beinhaltet aber auch Strategie und Taktik. Dies konnte noch nicht angeschaut werden und wird sicher ein Teil der Bachelorarbeit sein. Die Ameisen sollten sich in einem Schwarm fortbewegen, damit sie im Kampf st�rker sind. Dies zu implementieren w�re eine interessante Herausforderung in einem Kerngebiet der K�nstlichen Intelligenz.
Weiter kann gepr�ft werden, ob ein supervised oder ein unsupervised Learning eingebaut werden kann.

Die Organisatoren der AI Challenge sind allerdings bereits dabei, die n�chste Challenge zu planen. Falls sich das n�chste Spiel als geeignet f�r eine Bachelorarbeit erweist, w�rden wir es bevorzugen, an dieser Challenge teilzunehmen, da wir uns so auch direkter mit den anderen Teilnehmern messen k�nnten.

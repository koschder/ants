\newglossaryentry{Tile}{name={Tile},description={Ortsangabe auf dem Spielfeld mit Row (Zeile) und Column (Spalte) beschrieben. Bsp: <r:12 c:10>}}
\newglossaryentry{FogofWar}{name={Fog of War},description={Teil der Karte, der durch die eigenen Einheiten nicht mehr sichtbar ist}}
\newglossaryentry{AI}{name={AI},description={Artificial Intelligence - K�nstliche Intelligenz}}
\newglossaryentry{Bot}{name={Bot},description={AI-Agent f�r ein Computerspiel}}
\newglossaryentry{API}{name={API},description={Application Programming Interface - Programmierschnittstelle}}
\newglossaryentry{BFS}{name={BFS},description={Breadth First Search - Breitensuche}}
\newglossaryentry{InfluenceMap}{name={Influence Map},description={Datenstruktur, die zur Berechnung des Einflusses von Spieleinheiten auf die Spielkarte dient}}
\newglossaryentry{LazyInitialization}{name={Lazy Initialization},description={Bei der Lazy Initialization wird eine Ressource erst beim erstmaligen Gebrauch initialisiert (z.B. ein Logfile erst mit dem ersten Log-Eintrag erstellt)}}
\newglossaryentry{PairProgramming}{name={Pair Programming},description={Paarprogrammierung bedeutet, dass bei der Erstellung des Quellcodes jeweils zwei Programmierer an einem Rechner arbeiten. Ein Programmierer schreibt den Code, w�hrend der andere �ber die Problemstellungen nachdenkt, den geschriebenen Code kontrolliert sowie Probleme, die ihm dabei auffallen, sofort anspricht\footnote{Definition von \url{http://de.wikipedia.org/wiki/Paarprogrammierung}}}}
\newglossaryentry{UML}{name={UML},description={Unified Modeling Language}}
\newglossaryentry{Xathis}{name={Xathis},description={Siegerbot der Ants AI Challenge}}
\newglossaryentry{Greentea}{name={Greentea},description={Zweitbester Bot der Ants AI Challenge}}
\newglossaryentry{Egreavette}{name={Egreavette},description={Dieser Bot klassierte sich als 93ster im Schlussklassement und war f�r uns ein ebenb�rtiger Gegner}}
\newglossaryentry{Deprecated}{name={Deprecated},description={Als deprecated markierter Code ist veraltet und sollte nicht mehr verwendet werden}}
\newglossaryentry{{MinMax Algorithmus}}{name={MinMax Algorithmus},description={Algorithmus zur Ermittlung optimaler Spielstrategien}}
\newglossaryentry{AStar}{name={A*},description={Pfadsuchalgorithmus siehe \ref{subsec:module.Suchalgorithmen.Pfadsuche.Astar}}}
\newglossaryentry{AStar}{name={HPA*},description={Hierarcical Path A*, Pfadsuchalgorithmus siehe \ref{subsec:module.Suchalgorithmen.Pfadsuche.HPAstar}}}
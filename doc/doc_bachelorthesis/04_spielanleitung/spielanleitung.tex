\chapter{Spielanleitung}
\label{chap:spielanleitung}

Dieser Anhang beschreibt kurz, wie ein Spiel mit unserem Bot ausgef\"uhrt werden kann.
  
Das File \texttt{Ants.zip} enth\"alt das Eclipse-Projekt mit dem gesamten Source-Code unserer Implementation, und der offiziellen Spiel-Engine. Zum einfachen Ausf\"uhren eines Spiels haben wir ein ANT-Buildfile (build.xml) erstellt. Dieses definiert verschiedene Targets, mit denen ein Spiel mit jeweils unterschiedlichen Parametern gestartet werden kann.
\begin{enumerate}
\item
Das Target \texttt{testBot} ist lediglich zum einfachen Testen eines Bots sinnvoll und entspricht dem Spiel, das verwendet wird, um Bots, die auf der Website hochgeladen werden, zu testen.
\item
Das Target \texttt{runTutorial} f\"uhrt ein Spiel mit den Parametern aus, die im Tutorial auf der Website zur Erkl\"arung der Spielmechanik verwendet werden.
\item
Die Targets \texttt{mazeAgainst*} f\"uhren jeweils ein Spiel auf einer komplexeren und gr\"osseren, labyrinthartigen Karte aus, und zwar gegen den jeweils bezeichneten Gegner.
\item
Das Target \texttt{maze4Players} f�hrt ein Spiel gegen drei andere Bots auf einer noch etwas gr�sseren Karte durch.
\item
Die Targets \texttt{mazeProfiles} und \texttt{mazeProfiles4Players} sind dazu da, mehrere Kopien unseres Bots mit verschiedenen Profilen gegeneinander antreten zu lassen. Wie im Kapitel \ref{sec:module.Logging.Profile} erw�hnt, muss dazu erst in der Klasse LiveInfo das LiveInfo-Logging deaktiviert werden: 
\lstset{language=Java, tabsize=4}
\begin{lstlisting}
private static boolean enabled = false;
\end{lstlisting}
\item 
Das Target \texttt{testOnline} l�sst den Bot �ber einen privaten TCP-Server gegen andere Bots antreten , unter �hnlichen Bedingungen wie beim eigentlichen Wettbewerb, nur leider mit deutlich weniger Teilnehmern. (s. Kapitel \ref{sec:testCenter.Online})
\item
Die Targets \texttt{repeat*} dienten der Durchf�hrung der Profil-Testl�ufe und sind ansonsten wenig interessant.
\item
Die Target \texttt{debug*} dienten dem Starten unserer Testbots, die wir w�hrend der Entwicklung einsetzten (s. Kapitel \ref{sec:testCenter.Testbots}).
\end{enumerate}

Bei den meisten dieser Targets erscheint beim Starten eine Abfrage, welches Profil man gerne starten w�rde. Hier stehen die in Kapitel \ref{sec:module.Profile.DefinierteProfile} beschriebenen Profile zur Auswahl.

Im Unterordner \texttt{tools} befindet sich die in Python implementierte Spiel-Engine. Unter \texttt{tools/maps} liegen noch weitere vordefinierte Umgebungen, und unter \texttt{tools/mapgen} liegen verschieden Map-Generatoren, die zur Erzeugung beliebiger weiterer Karten verwendet werden k\"onnen. 

Im Unterordner \texttt{tools/sample\_bots} befinden sich einige einfache Beispiel-Bots, gegen die man spielen kann. 

Im Unterordner \texttt{tools/bots} haben wir Kopien der Bots von einigen Teilnehmern abgelegt, gegen die wir im Zug der Entwicklung immer wieder getestet haben. 
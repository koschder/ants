\section{Missionen}
\label{sec:module.Missionen}
\begin{figure}[bth]
\centering
\includegraphics[width=0.9\textwidth]{91_bilder/Missions}
\caption{Missionen}
\label{fig:missions}
\end{figure}

TODO NEW

Eine Mission dauert �ber mehrere Spielz�ge. Die meisten Missionen (GatherFoodMission, ExploreMission, AttackHillMission, AttackAntMission) sind Pfadmissionen\footnote{Die abstrakte Klasse PathMission ist im Code unter ants.missions.PathMission.java auffindbar.}, bei welchen die Ameise einem vorgegebenen Pfad, der bereits beim Erstellen der Mission berechnet wurde, folgt. 
Die FollowMission ist eine spezielle Mission, mit der eine Ameise einfach einer anderen Ameise hinterherl�uft.

Eine Mission kann auch abgebrochen werden, wenn es keinen Sinn mehr macht, sie weiter zu verfolgen. Je nach spezifischer Mission sind aber die Abbruchbedingungen anders. Zum Beispiel die GatherFoodMission ist nur solange g�ltig wie das Futter noch nicht von einer anderen Ameise eingesammelt wurde.
Abbildung \ref{fig:missions} zeigt einen \"Uberblick �ber die wichtigsten Missionen und ihre Hierarchie.




F�r jedes Food-Tile werden in einem definierbaren Radius r die n�chsten Ameisen bestimmt. Danach wird nach aufsteigender Luftliniendistanz mit dem Pfadsuchalgorithmus SIMPLE (s. Abschnitt \ref{subsec:implementation.Pfadsuche.Simple}) oder -- falls dieser keinen Pfad gefunden hat -- mit A* eine passierbare Route gesucht. Wenn ein Pfad existiert, kann mit der Ameise und dem Food-Tile eine GatherFoodMission erstellt werden, welche die Ameise zum Food-Tile f�hrt. Zu jedem Food-Tile wird immer nur eine Ameise geschickt.
\section{Breitensuche}
\label{subsec:module.Suchalgorithmen.Breitensuche}

Die Breitensuche (engl. breadth-first search (BFS)) war eine der Neuimplementierungen w�hrend der Bachelorarbeit. Wir verwenden diese Suche um die Umgebung einer Ameise oder eines H�gels nach Futter, Gegnern usw. zu scannen. Man k�nnte die BFS auch f�r die Pfadsuche verwenden, dies w�re aber sehr ineffizient. Im Klassendiagramm ist zu sehen auf welchen drei Methoden die Breitensuche aufbaut.

\begin{figure}[H]
\centering
\includegraphics[width=0.7\textwidth]{91_bilder/BFS}
\caption{Breitensuche Klassendiagramm}
\label{fig:BFS}
\end{figure}

TODO FRONTIER TEST fehlt auf dem Bild


Die Breitensuche wurde generisch implementierte, so dass sie vielseitig einsetzbar ist. So k�nnen zum Beispiel mittels 'GoalTest' je nach Anwendungsfall die Tiles beschrieben werden die gesucht sind. Folgende Breitensuche findet die Ameise welche am n�chsten bei einem Food-Tile <r:20,c:16> ist. Die Suche wird initialisiert indem im Konstruktor die Spielkarte mitgegeben wird, welche durchforscht wird. Zus�tzlich gilt die Einschr�nkung das die Breitensuche nur 40 Tiles durchsuchen darf, was einem Radius von zirka 7 Zellen entspricht. Falls keine Ameise gefunden wird gibt der Algorithmus NULL zur�ck.

\lstset{language=Java, tabsize=4}
\begin{lstlisting}
AntsBreadthFirstSearch bfs = new AntsBreadthFirstSearch(Ants.getWorld());
Tile food = new Tile(20,16);
Tile antClosestToFood = bfs.findSingleClosestTile(food, 40, new GoalTest() {
      @Override
      public boolean isGoal(Tile tile) {
          return isAntOnTile(tile);
      }
  });
\end{lstlisting}

Es ist auch m�glich mehrere Tiles zur�ck zu bekommen. Dazu wird die Methode \textit{findClosestTiles(...)} aufgerufen.\\
\\
Der gleiche Algorithmus kann aber auch alle passierbaren Tiles in einem gewissen Umkreis zur�ckgeben. Dies haben wir unter anderem beim Initialisieren der DefendHillMission verwendet. Wir berechnen beim Erstellen der Mission die passierbaren Zellen rundum den H�gel. Runde f�r Runde pr�fen wir diese Tiles auf gegnerische Ameisen um die entsprechenden Verteidigungsmassnahmen zu ergreifen. Der Parameter controlAreaRadius2 definiert den Radius des 'Radars' und kann je nach Profile unterschiedlich eingestellt werden.

\lstset{language=Java, tabsize=4}
\begin{lstlisting}
public DefendHillMission(Tile myhill) {
    this.hill = myhill;
    BreadthFirstSearch bfs = new BreadthFirstSearch(Ants.getWorld());
    tilesAroundHill = bfs.floodFill(myhill, controlAreaRadius2);
}
\end{lstlisting}


Um die Aufrufe der Suche im Ants-Umfeld einfacher zu gestalten haben wir die Breitensuche f�r unseren Bot mit folgenden selbst-sprechenden Methoden erweitert.

\begin{figure}[H]
\centering
\includegraphics[width=0.5\textwidth]{91_bilder/BFSants}
\caption{Breitensuche Ants-spezifisch}
\label{fig:BFSants}
\end{figure}



\subsection{Barrier (Sperre)}
\label{subsec:module.Suchalgorithmen.Breitensuche.Barrier}

Eine Erweiterung der Breitensuche erm�glicht uns eine Sperre in der Umgebung eines Ortes zu finden. Diese Verwenden wir in der DefendHillMission zum Verteidigen des eigenen H�gels. Es kann nur eine Sperre (engl. Barrier) gefunden werden wenn das Gel�nde dazu passt. Die Abbildung zeigt einen gefundene Sperre. Auf dieser H�he wird der H�gel verteidigt.

\begin{figure}[H]
\centering
\includegraphics[width=0.5\textwidth]{91_bilder/barrier}
\caption{Auf der orangen Sperre werden die Ameisen zur Verteidigung des H�gel positioniert.}
\label{fig:search.barrier}
\end{figure}

Der Algorithmus verbirgt sich in der Methode \textit{getBarrier(...)}. Diese wird mit den Parametern \textit{tileToProtect}: Ort der durch eine Sperre gesch�tzt werden soll, \textit{viewRadiusSquared}: den Sichtradius der Einheiten, den die Sperre soll weiter entfernt sein als der Sichtradius, damit die gegnerischen Einheiten nicht sehen was sich dahinter verbirgt. Der dritte Parameter \textit{maximumBarrierSize} definiert welche Breite die Sperre maximal haben darf.

\lstset{language=Java, tabsize=4}
\begin{lstlisting}
public Barrier getBarrier(final Tile tileToProtect, int viewRadiusSquared, int maximumBarrierSize) {
	int amount = BFS for getting the amount of tiles in view radius around the location to defend.
	Barrier smallestBarrier = null;
	List<Tile> tiles = get (amount + 30) tiles around the location to defend.
	
	// for loop start at the first tile not in view radius
	for(int i = amount;i<tiles.size(); i++){       
		Tile t = tiles.get(i);
		
		//vertical check
		if(!barrierVerticalInvalid.contains(t)){
			Barrier b = get vertical barrier on position of Tile t
			if(b is smaller than 5 Tiles && smaller than smallestBarrier){
				if(is barrier the only exit out of the location to defend){
						smallestBarrier = b;
				}else{
						 //add all tiles of the invaild barrier
						barrierVerticalInvalid.add(b.getTiles());
				}    					      		
			}else{
				 //add all tiles of the invaild barrier
				barrierVerticalInvalid.add(b.getTiles());
			}
		}
		
		//horizontal check
		if(!barrierHorizontalInvalid.contains(t)){
			Barrier b = get horiontal barrier on position of Tile t
			if(b is smaller than 5 Tiles && smaller than smallestBarrier){
				if(is barrier the only exit out of the location to defend){
						smallestBarrier = b;
				}else{
						 //add all tiles of the invaild barrier
						barrierHorizontalInvalid.add(b.getTiles());
				}    					      		
			}else{
				//add all tiles of the invaild barrier
				barrierHorizontalInvalid.add(b.getTiles());
			}
		}	
	}
}
\end{lstlisting}

Dank dem Abspeichern der ung�ltigen Tiles aller zu breiten Sperren in die Listen \textit{barrierHorizontalInvalid} und \textit{barrierVerticalInvalid} konnte der Algorithmus markant schneller gemacht werden. F�r diese Tiles muss nicht nochmals eine Sperre berechnet werden. Auch die if-Abfrage \textit{barrier is the only exit out of the location to defend} muss nicht mehr oft aufgerufen werden, den hinter dieser Abfrage steht n�mlich wiederum ein Test mit der Breitensuche. Dieser zus�tzliche Test mit der Breitensuche ist viel teurer als das Zwischenspeichern der Tiles aus welchen keine g�ltige Sperre gemacht werden konnte.\\
\\
Im Nachhinein hat sich ergeben, dass nicht unbedingt die schmalste Sperre die Beste w�re, sondern jede bei welche der Gegner, gel�ndebedingt, weniger Einheiten aufstellen kann. So w�re in Abbildung \ref{fig:search.barrier} eine Sperre zwei Zellen �stlicher besser, den der Gegner k�nnte beim Angriff nur mit vier Einheiten vorr�cken, die Verteidigung w�re aber mit sechs Einheiten auf einer Linie deutlich st�rker. Die Zeit hat aber hier leider nicht gereicht, den Algorithmus weiter zu verfeinern.
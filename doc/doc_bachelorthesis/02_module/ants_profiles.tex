\section{Profile}
\label{sec:module.Profile}
\"Uber die Profile l�sst sich das Verhalten unseres Bots konfigurativ beeinflussen. Folgende Parameter stehen dazu zur Verf�gung:
\begin{itemize}
\item
\textbf{defaultAllocation.gatherFood}: Standard Ressourcenzuteilung in Prozent f�r die Nahrungssuche. Dieser Wert dient als Startwert f�r den ResourceAllocator.
\item
\textbf{defaultAllocation.explore}:Standard Ressourcenzuteilung in Prozent f�r die Erkundung der Spielwelt. Dieser Wert dient als Startwert f�r den ResourceAllocator.
\item
\textbf{defaultAllocation.attackHills}:Standard Ressourcenzuteilung in Prozent f�r den Angriff auf gegnerische H�gel. Dieser Wert dient als Startwert f�r den ResourceAllocator.
\item
\textbf{defaultAllocation.defendHills}:Standard Ressourcenzuteilung in Prozent f�r die Verteidigung der eigenen H�gel. Dieser Wert dient als Startwert f�r den ResourceAllocator.
\item
\textbf{defendHills.startTurn}: In der Anfangsphase des Spiels macht es nicht viel Sinn, Verteidiger zur�ckzubehalten obwohl kein Angreifer in der N�he ist. Mit diesem Parameter kann eingestellt werden, ab welchem Zug wir pro H�gel mindestens einen Verteidiger abkommandieren.
\item
\textbf{defendHills.minControlRadius}: Dieser Parameter legt fest, wie gross das Gebiet um unsere H�gel ist, das wir verteidigen.
\item
\textbf{explore.forceThresholdPercent}: Schwellenwert f�r das Verh�ltnis von nie gesehenen Zellen zu allen Zellen auf der Karte. Wenn der Wert unter diesem Schwellenwert liegt, werden zus�tzliche Ressourcen f�r die Erkundung bereitgestellt.
\item
\textbf{explore.forceGain}: Dieser Wert bestimmt, um wie viel die Ressourcen f�r die Erkundung erh�ht werden, falls obiger Schwellenwert unterboten wird. M�gliche Werte liegen zwischen 0 und 1.
\item
\textbf{explore.dominantPositionBoost}: Wenn wir eine dominante Position auf dem Spielfeld haben, wird die Erkundung gest�rkt, um die Karte zu kontrollieren. Dieser Parameter bestimmt, wie viel zus�tzliche Ressourcen zugeteilt werden.
\item
\textbf{attackHills.dominantPositionBoost}: Wenn wir eine dominante Position auf dem Spielfeld haben, wird der Angriff auf gegnerische H�gel forciert, um Punkte zu sammeln. Dieser Parameter bestimmt, wie viel zus�tzliche Ressourcen zugeteilt werden.
\item
\textbf{attackHills.halfTimeBoost}: Nach Ablauf der H�lfte der Spielzeit erh�hen wir die Aggressivit�t unseres Bots. Dieser Parameter bestimmt, wie stark der Angriff auf gegnerische H�gel in der 2. Halbzeit forciert wird.
\end{itemize}

\subsection{Validierung}
\label{sec:module.Profile.Validierung}
Die Parameter \texttt{defaultAllocation.*} m�ssen in der Summe 100 ergeben. Dies wird beim Start des Bots, wenn das Profil geladen wird, �berpr�ft. Auch f�r die anderen Parameter sind Validierungen vorhanden (z.B. Pr�fung, ob ein Prozentwert zwischen 0 und 100 liegt.)

Fehlt ein Parameter in einem Profil, so wird f�r diesen Parameter der Standardwert genommen.

\subsection{Definierte Profile}
\label{sec:module.Profile.DefinierteProfile}
Folgende Profile sind im Bot definiert und k�nnen beim Start ausgew�hlt werden:

\renewcommand{\arraystretch}{1.5}
		\begin{tabular}{ l | r | r | r | r }
			Parameter & Default & Aggressive & Defensive & Expansive \\
			\hline
			defaultAllocation.gatherFood & 25 & 20 & 25 & 30\\
			defaultAllocation.explore & 25 & 20 & 25 & 30\\
			defaultAllocation.attackHills & 25 & 45 & 15 & 20\\
			defaultAllocation.defendHills & 25 & 15 & 35 & 20\\
			defendHills.startTurn & 30 & 30 & 20 & 30\\
			defendHills.minControlRadius & 8 & 8 & 12 & 8\\
			explore.forceThresholdPercent & 80 & 80 & 80 & 90\\
			explore.forceGain & 0.25 & 0.25 & 0.25 & 0.4\\
			explore.dominantPositionBoost & 5 & 5 & 5 & 8\\
			attackHills.dominantPositionBoost & 2 & 5 & 2 & 2\\
			attackHills.halfTimeBoost & 20 & 25 & 15 & 20\\
		\end{tabular}



\subsection{Weiterentwicklungs-Potenzial}
\label{sec:module.Profile.Weiterentwicklung}
Aktuell k�nnen 11 Parameter via Profil angepasst werden, wovon die meisten einen Einfluss auf das Ressourcen-Management haben. Weitere Parameter f�r beliebige Teile der Logik k�nnten problemlos eingef�gt werden und m�ssten sich auch nicht auf numerische Werte beschr�nken. Denkbar w�ren bspw. ganze Strategien, die per Profil einfach ausgetauscht werden k�nnten.
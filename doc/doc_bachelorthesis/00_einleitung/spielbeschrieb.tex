\section{Spielbeschrieb}
\label{sec:einleitung.Spielbeschrieb}

\subsection{Der Wettbewerb}
\label{sec:einleitung.Spielbeschrieb.wettbewerb}
Die AI Challenge ist ein internationaler Wettbwerb des University of Waterloo Computer Science Club, der im Zeitraum Herbst 2011 bis Januar 2012 zum 3. Mal stattgefunden hat. Das Spiel in dieser 3. Ausf�hrung war ein zugbasiertes Multiplayerspiel, in welchem sich Ameisenv�lker gegenseitig bek�mpfen. Ziel in der AI-Challenge ist es, einen Bot zu schreiben, der die gegebenen Aufgaben mit m�glichst intelligenten Algorithmen l�st. Die zu l�senden Aufgaben der Ants AI Challenge sind die Futtersuche, das Explorieren der Karten, das Angreifen von gegnerischen V�lkern und deren Ameisenhaufen sowie dem Verteidigen des eigenen Ameisenhaufen.

\subsection{Spielregeln}
\label{sec:einleitung.Spielbeschrieb.spielregeln}
Nachfolgend sind die wichtigsten Regeln, die w�hrend dem Spiel zu ber�cksichtigen sind, aufgelistet.
\begin{itemize} 
\item Pro Zug k�nnen alle Ameisen um ein Feld (vertikal oder horizontal) verschoben werden.
\item Pro Zug steht insgesamt eine Rechenzeit von einer Sekunde zur Verf�gung. Es d�rfen keine Threads erstellt werden.
\item Bewegt sich eine Ameise in die 4er Nachbarschaft einer Futterzelle, wird das Futter eingesammelt. Daraufhin schl�pft im n�chsten Zug eine neue Ameise aus einem der Ameisenh�gel.
\item Die Landkarte besteht aus passierbaren Landzellen sowie unpassierbaren Wasserstellen.
\item Ein Gegner wird geschlagen, wenn im Kampfradius der eigenen Ameise mehr eigene Ameisen stehen als gegnerische Ameisen im Kampfradius der Ameise, die angegriffen wird.
\item Ein Gegner ist ausgeschieden wenn alle seine eigenen Ameisenh�gel vom Gegner vernichtet wurden. Pro verlorenem H�gel gib es einen Punkteabzug. Pro zerst�rten feindlichen H�gel gibt es zwei Bonuspunkte.
\item Steht nach einer definierbaren Zeit (Anzahl Z�ge) kein Sieger fest, wird der Sieger anhand der erreichten Punkte ermittelt. 
\end{itemize}
Die ausf�hrlichen Regeln k�nnen auf der Webseite nachgelesen werden: \url{http://aichallenge.org/specification.php}

\subsection{Schnittstelle}
\label{sec:einleitung.Spielbeschrieb.schnittstelle}
Die Spielschnittstelle ist simpel gehalten. Nach jeder Spielrunde erh�lt der Bot das neue Spielfeld mittels String-InputStream. Die Spielz�ge gibt der Bot dem Spielcontroller mittels String-OutputStream bekannt. Unser MyBot leitet von der Basis-Klasse Bot ab, die Teil des Starter-Pakets ist, das von den Wettbewerbs-Organisatoren zur Verf�gung gestellt wird. Ein Spielzug wird im folgendem Format in den Output-Stream gelegt:
\begin{lstlisting}
o <Zeile> <Spalte> <Richtung>
\end{lstlisting}
Beispiel:
\begin{lstlisting}
o 4 7 W
\end{lstlisting}
Die Ameise wird von der Position Zeile 4 und Spalte 7 nach Westen bewegt.

Der Spielcontroller ist in Python realisiert, der Bot kann aber in allen g�ngigen Programmiersprachen wie Java, Python, C\#, C++ etc. geschrieben werden.

\section{Fazit}
\label{sec:einleitung.Fazit}
Alles in allem sind wir mit dem Verlauf der Arbeit zufrieden. Wir konnten unsere Ziele erreichen und konnten dank einer strukturierten Arbeitsweise auch den Zeitplan gr�sstenteils einhalten. 
In Bezug auf die Wettbewerbsf�higkeit des Bots hatten wir uns bewusst keine konkreten Ziele gesetzt, da wir wussten, dass es in der beschr�nkten Zeit sehr schwierig werden w�rde, mit den Top 100 des Wettbewerbs mitzuhalten. Insofern sind wir auch mit dem Ergebnis zufrieden: Gegen den Bot auf Platz 93 konnten wir rund 60\% der Spiele gewinnen, und gegen den Sieger des Wettbewerbs immerhin um die 20\% der Spiele. (Details s. Kapitel \ref{sec:testCenter.TestreportProfile})

Besonders positiv hervorheben m�chten wir die Arbeitsprozesse und Tools, die uns diesen guten Projektverlauf erm�glicht haben. Die w�chentlichen Treffen mit \gls{PairProgramming} waren dabei besonders produktiv; unter der Woche konnten wir uns dank einer guten Arbeitsaufteilung und der Unterst�tzung durch die Versionsverwaltung jeweils auf das Wesentliche konzentrieren, ohne durch administrative oder technische Probleme abgelenkt zu werden.

Ausserdem war nat�rlich die Arbeit am Bot sehr spannend, da wir hier die Gelegenheit hatten, den in den letzten 8 (und insbesondere den letzten 3) Semestern behandelten Stoff in die Praxis umzusetzen. Besonders interessant war dabei die Herausforderung, die einzelnen, isolierten Techniken, die wir gelernt hatten, zu einem strukturierten Ganzen zusammenzuf�gen.

Ein kleiner Wermutstropfen bleibt nat�rlich, dass wir unseren Bot nicht unter Wettbewerbsbedingungen testen konnten; die Testl�ufe gegen die verschiedenen Gegner erlaubten uns zwar eine ungef�hre Standortbestimmung, aber das ist nat�rliche nicht das gleiche. Auch sind wir, wie bereits erw�hnt, nicht ganz zufrieden mit der Implementierung des Kampfverhaltens unserer Ameisen. Mit etwas mehr Zeit h�tten wir da gerne noch eine genauere und dynamischere Bewertung von Kampfsituationen eingebaut.
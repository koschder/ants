\section{Weiterf�hrende Arbeiten}
\label{sec:einleitung.WeiterfuehrendeArbeiten}

Obwohl wir mit dieser Arbeit grunds�tzlich unsere Ziele erreicht haben, gibt es nach wie vor Verbesserungspotenzial in unserem Bot. Die wichtigsten davon sind folgende:

\paragraph{CombatPositioning}
In unserem CombatPositioning haben wir zwar eine ganz gute N�herung an eine optimale Formation gefunden, aber im Kampf gegen die besten Bots zeigte sich doch deutlich, dass diese noch nicht gut genug ist. Hier k�nnten wir das grunds�tzliche Code-Design zwar gut beibehalten, aber die Logik f�r die Positionierung der Einheiten m�sste noch verbessert werden.

\paragraph{Profile}
Die Profile haben wir erst relativ sp�t eingef�hrt und uns deshalb etwas eingeschr�nkt im Bezug darauf, was alles konfigurierbar ist. Mit etwas Fleissarbeit k�nnte man diese Profile noch stark erweitern; dies k�nnte man sogar so weit treiben, dass aus dem Verhalten zweier unterschiedlich konfigurierter Bots kaum mehr erkennbar w�re, dass sie die gleiche Codebasis haben.

\paragraph{Regeln f�r Ressourcenverwaltung}
Ebenfalls eine Komponente, mit der sich die ''Pers�nlichkeit`` unseres Bots noch flexibler gestalten liesse: Mit der Implementierung von weiteren Regeln f�r die Ressourcenverwaltung k�nnte die Umsetzung von strategischen Entscheidungen noch dynamischer gemacht werden.

\paragraph{Weitere Challenges}
Und falls die Organisatoren der AI-Challenge doch noch einmal ein neues Spiel lancieren, k�nnten wir vermutlich dank des modularen Aufbaus unseres Codes einige Komponenten f�r dieses neue Spiel wiederverwenden; welche genau, h�ngt nat�rlich vom Spiel ab.

\section{Projektorganisation}
\label{sec:einleitung.Projektorganisation}

\subsection{Beteiligte Personen}
\label{sec:einleitung.Projektorganisation.BetroffenePersonen}

\textbf{Studierende:}\\
 Lukas Kuster \textit{kustl1@bfh.ch} \\
 Stefan K�ser \textit{kases1@bfh.ch}
 
\textbf{Betreuung:}\\
Dr. J�rgen Eckerle \textit{juergen.eckerle@bfh.ch}

\textbf{Experte:}\\
Dr. Federico Fl�ckiger	\textit{federico.flueckiger@bluewin.ch}


\subsection{Projektmeetings}
\label{sec:einleitung.Projektorganisation.Projektmeetings}

\begin{itemize}
\item Es fand jeweils ein Treffen mit dem Betreuer alle 1-2 Wochen statt.
\item Ein Treffen mit dem Experten fand am Anfang der Arbeit statt. Ein zweites Meeting wurde von beiden Seiten als nicht n�tig gehalten.
\end{itemize}

\subsection{Dokumentation}
\label{sec:einleitung.Projektorganisation.Dokumentation}

Die Dokumentation soll sich am Aufbau und Inhalt des Berichts aus dem Projekt 2 anlehnen.
\begin{itemize}
\item Das Dokument beschr�nkt sich auf das Wesentliche.
\item Verwendete AI-Techniken werden erl�utert
\item Entscheidungen und deren Grundlagen sind dokumentiert.
\item Testberichte dokumentieren die durchgef�hrten Modultests. 
\item Klassendiagramme sollen einen oberfl�chlichen Detailierungsgrad haben, so dass das Wichtigste auf den ersten Blick sichtbar ist.
\item Anleitung zum Ausf�hren eines Spiels
\end{itemize}

\subsection{Abgabe}
\label{sec:einleitung.Projektorganisation.Abgabe}
Folgende Lieferobjekte werden am Ende der Arbeit abgegeben.
\begin{itemize}
\item Dokumentation
\item Sourcecode
\end{itemize}

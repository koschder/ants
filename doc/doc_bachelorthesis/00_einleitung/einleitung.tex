\chapter{Einleitung}
\label{chap:einleitung}
Nachdem wir uns im Rahmen des Moduls ''Projekt 2`` (7302) mit der Implementierung eines Bots f�r den Online-Wettbewerb AI-Challenge (Ants) besch�ftigt hatten, haben wir uns f�r die Bachelorarbeit eine Verbesserung dieses Bots vorgenommen. Die AI-Challenge ist ein Wettbewerb, der im Herbst 2011 zum 3. Mal stattfand und jedes Jahr mit einem anderen Spiel durchgef�hrt wird. Ziel ist es jeweils, einen Bot zu programmieren, der durch geschickten Einsatz von KI-Technologien das Spiel m�glichst erfolgreich bestreiten kann. In dieser Durchf�hrung ging es darum, ein Ameisenvolk durch Sammeln von Ressourcen und Erobern von gegnerischen H�geln zum Sieg �ber die gegnerischen Ameisen zu f�hren.

Im ''Projekt 2`` hatten wir zwar einen Bot implementiert, der alle Aspekte des Spiels einigermassen beherrscht, also Nahrung sammeln, die Gegend entdecken, H�gel erobern und verteidigen, sowie gegen feindliche Ameisen k�mpfen. Einige dieser F�higkeiten waren aber eher rudiment�r ausgebaut, da wir uns  vor allem auf die Pfadsuche konzentriert hatten.

In der Bachelorarbeit ging es nun darum, die taktischen und strategischen Fertigkeiten des Bots auszubauen. Der Schwerpunkt lag auch bei der Bachelorarbeit nicht auf der Optimierung einer Teilaufgabe, sondern auf der Implementierung eines ausgewogenen Bots, der alle Aspekte des Spiels gleichermassen gut beherrscht.

Besonderes Augenmerk legten wir dabei auf einen modularen Aufbau des Codes. Nebst einem sauberen objektorientierten Programmdesign spiegelt sich das vor allem in den separaten Modulen ''AITools-Api``, ''Search`` und ''Strategy``, die so generisch implementiert wurden, dass sie mit geringem Aufwand auch in anderen Projekten einsetzbar sind.
\newpage


\section{Herausforderungen}
\label{sec:einleitung.Herausforderungen}

\subsection{Module testen}
\label{sec:einleitung.Herausforderungen.ModuleTesten}

Ein neuer Algorithmus oder eine neue Idee ist schnell mal in den Bot integriert, doch bringen die geschriebenen Zeilen den gew�nschten Erfolg? Was wenn der neue Codeabschnitt �usserst selten durchlaufen wird und dann noch fehlschl�gt? Wie wissen wir welche Ameise genau diesen n�chsten Schritt macht?

Um diese Probleme zu bew�litgen haben wir ein ausgekl�geltes Logging auf die Beine gestellt, in welchem wir schnell an die gew�nschten Informationen gelangen. (siehe \ref{sec:module.Logging})  Zudem k�nnen wir dank der Erweiterung des HTML-Viewer sofort sehen, welches die akutelle Aufgabe jeder einzelen Ameise ist. (siehe \ref{sec:module.Logging.Addon}) Weitergeholfen haben uns auch etliche Unit- und Funktionstests, mit welchen wir neu geschriebenen Code testen und auf dessen Richtigkeit pr�fen konnten. (siehe \ref{sec:testCenter.UnitundFunktionstests})

\subsection{TODO}
\label{sec:einleitung.Herausforderungen.TODO}
lorem ipsum mehr herausfoderungen ??

\subsection{Vergleich mit Bots aus dem Wettbewerb}
\label{sec:einleitung.Herausforderungen.VergleichBots}
Nach Ablauf des Wettbewerbs im Januar 2012, haben einige der Teilnehmer ihren Bot zug�nglich gemacht. Dadurch war es uns m�glich unseren Bot gegen Bots antreten zu lassen die tats�chlich am Wettbewerb teilgenommen haben. So konnten wir auch eine wage Einsch�tzung machen wir stark unser Bot ist. Mehr dazu unter siehe \ref{sec:testCenter.TestreportProfile}.
\section{Herausforderungen}
\label{sec:einleitung.Herausforderungen}
An dieser Stelle sollen einige Herausforderungen, die wir w�hrend der Arbeit angetroffen haben, besonders hervorgehoben werden.

\subsection{Module testen}
\label{sec:einleitung.Herausforderungen.ModuleTesten}

Ein neuer Algorithmus oder eine neue Idee ist schnell mal in den Bot integriert, doch bringen die geschriebenen Zeilen den gew�nschten Erfolg? Was wenn der neue Codeabschnitt �usserst selten durchlaufen wird und dann noch fehlschl�gt? Wie wissen wir welche Ameise genau diesen n�chsten Schritt macht?

Um diese Probleme zu bew�litgen haben wir ein ausgekl�geltes Logging auf die Beine gestellt, in welchem wir schnell an die gew�nschten Informationen gelangen. (siehe Kapitel \ref{sec:module.Logging})  Zudem k�nnen wir dank der Erweiterung des HTML-Viewer sofort sehen, welches die akutelle Aufgabe jeder einzelen Ameise ist. (siehe \ref{sec:module.Logging.Addon}) Weitergeholfen haben uns auch etliche Unittests und visuelle Tests, mit welchen wir neu geschriebenen Code testen und auf dessen Richtigkeit pr�fen konnten. (siehe Kapitel \ref{sec:testCenter.UniTests} und \ref{sec:testCenter.VisuelleTests})

\subsection{Kampfsituationen beurteilen}
\label{sec:einleitung.Herausforderungen.Kampfsituationen}
Die Beurteilung von Kampfsituationen stellte sich wie erwartet als Knacknuss heraus. In einem ersten Anlauf versuchten wir eine gegebene Kampfsituation mittels MinMax Algorithmus zu beurteilen. Wir hatten bereits aus einem fr�heren Projekt im Rahmen des Moduls Spieltheorie (7501) eine Implementation des Algorithmus erstellt und hatten eigentlich auch keine M�he, diese f�r unsere Ameisen anzupassen. Aufgrund des hohen Branching-Faktors einer Kampfsituation (eine Ameise hat 5 m�gliche Z�ge, und an einer Kampfsituation sind i.A. 5-10 Ameisen beteiligt) konnten wir die n�tigen Berechnungen aber nicht in der zur Verf�gung stehenden Zeit durchf�hren.
Die aktuelle Implementierung der Kampfsituationen verfolgt jetzt einen weniger rechenintensiven Ansatz; im Gegensatz zum MinMax Algorithmus kann sie aber keine optimale L�sung finden.


\subsection{Vergleich mit Bots aus dem Wettbewerb}
\label{sec:einleitung.Herausforderungen.VergleichBots}
Nach Ablauf des Wettbewerbs im Januar 2012, haben einige der Teilnehmer ihren Bot zug�nglich gemacht. Dadurch war es uns m�glich unseren Bot gegen Bots antreten zu lassen die tats�chlich am Wettbewerb teilgenommen haben. So konnten wir auch eine vage Einsch�tzung machen wir stark unser Bot ist. Mehr dazu unter siehe \ref{sec:testCenter.TestreportProfile}.
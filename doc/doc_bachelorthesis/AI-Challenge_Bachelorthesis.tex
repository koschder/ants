%
% Main document
% ===========================================================================
% This is part of the document "Project documentation template".
% Authors: brd3
%

%---------------------------------------------------------------------------
\documentclass[
	a4paper,					% paper format
	10pt,							% fontsize
	twoside,					% double-sided
	openright,				% begin new chapter on right side
	notitlepage,			% use no standard title page
	parskip=half,			% set paragraph skip to half of a line
]{scrreprt}					% KOMA-script report
%---------------------------------------------------------------------------

\raggedbottom
\KOMAoptions{cleardoublepage=plain}			% Add header and footer on blank pages


% Load Standard Packages:
%---------------------------------------------------------------------------
\usepackage[standard-baselineskips]{cmbright}

\usepackage[ngerman]{babel}										% german hyphenation
%\usepackage[latin1]{inputenc}  							% Unix/Linux - load extended character set (ISO 8859-1)
\usepackage[ansinew]{inputenc}  							% Windows - load extended character set (ISO 8859-1)
\usepackage[T1]{fontenc}											% hyphenation of words with �,� and �
\usepackage{textcomp}													% additional symbols
\usepackage{ae}																% better resolution of Type1-Fonts 
\usepackage{fancyhdr}													% simple manipulation of header and footer 
\usepackage{graphicx}                      		% integration of images
\usepackage{float}														% floating objects
\usepackage{caption}													% for captions of figures and tables
\usepackage{booktabs}													% package for nicer tables
\usepackage{tocvsec2}													% provides means of controlling the sectional numbering
\usepackage{color}
\usepackage{fancyvrb}
\usepackage{listings}
 

%---------------------------------------------------------------------------

% Load Math Packages
%---------------------------------------------------------------------------
\usepackage{amsmath}                    	   	% various features to facilitate writing math formulas
\usepackage{amsthm}                       	 	% enhanced version of latex's newtheorem
\usepackage{amsfonts}                      		% set of miscellaneous TeX fonts that augment the standard CM
\usepackage{amssymb}													% mathematical special characters
\usepackage{exscale}													% mathematical size corresponds to textsize
%---------------------------------------------------------------------------

% Package to facilitate placement of boxes at absolute positions
%---------------------------------------------------------------------------
\usepackage[absolute]{textpos}
\setlength{\TPHorizModule}{1mm}
\setlength{\TPVertModule}{1mm}
%---------------------------------------------------------------------------					
			
% Definition of Colors
%---------------------------------------------------------------------------
\RequirePackage{color}                          % Color (not xcolor!)
\definecolor{linkblue}{rgb}{0,0,0.8}            % Standard
\definecolor{darkblue}{rgb}{0,0.08,0.45}        % Dark blue
\definecolor{brickred}{cmyk}{0,0.89,0.94,0.28}  % Brickred
%\definecolor{linkcolor}{rgb}{0,0,0.8}     			% Blue for the web- and cd-version!
\definecolor{linkcolor}{rgb}{0,0,0}        			% Black for the print-version!
\definecolor{bfhred}{rgb}{0.776,0,0.066}  			% Red
\definecolor{commentgreen}{rgb}{0,0.5,0}  % Green for java comments
%---------------------------------------------------------------------------

% Hyperref Package (Create links in a pdf)
%---------------------------------------------------------------------------
\usepackage[
	pdftex,ngerman,bookmarks,plainpages=false,pdfpagelabels,
	backref = {false},										% No index backreference
	colorlinks = {true},                  % Color links in a PDF
	hypertexnames = {true},               % no failures "same page(i)"
	bookmarksopen = {true},               % opens the bar on the left side
	bookmarksopenlevel = {0},             % depth of opened bookmarks
	pdftitle = {Bachelor Thesis - AI Bot f�r Computerspiele},	   	% PDF-property
	pdfauthor = {kases1 kustl1},        					  % PDF-property
	pdfsubject = {Documentation},        % PDF-property
	linkcolor = {linkcolor},              % Color of Links
	citecolor = {linkcolor},              % Color of Cite-Links
	urlcolor = {linkcolor},               % Color of URLs
]{hyperref}
%---------------------------------------------------------------------------
 
\lstset{fancyvrb=true}
\lstset{
basicstyle=\small\tt,
keywordstyle=\color{darkblue},
identifierstyle=,
commentstyle=\color{commentgreen},
stringstyle=\color{brickred},
showstringspaces=false,
tabsize=3,
captionpos=b,
numberstyle=\tiny,
breaklines=true
%stepnumber=4
}
% Set up page dimension
%---------------------------------------------------------------------------
\usepackage{geometry}
\geometry{
	a4paper,
	left=28mm,
	right=15mm,
	top=30mm,
	headheight=20mm,
	headsep=10mm,
	textheight=242mm,
	footskip=15mm
}
%---------------------------------------------------------------------------

% Makeindex Package
%---------------------------------------------------------------------------
\usepackage{makeidx}                         		% To produce index
\makeindex                                    	% Index-Initialisation
%---------------------------------------------------------------------------

% Glossary Package
%---------------------------------------------------------------------------
% the glossaries package uses makeindex
% if you use TeXnicCenter do the following steps:
%  - Goto "Ausgabeprofile definieren" (ctrl + F7)
%  - Select the profile "LaTeX => PDF"
%  - Add in register "Nachbearbeitung" a new "Postprozessoren" point named Glossar
%  - Select makeindex.exe in the field "Anwendung" ( ..\MiKTeX x.x\miktex\bin\makeindex.exe )
%  - Add this [ -s "%tm.ist" -t "%tm.glg" -o "%tm.gls" "%tm.glo" ] in the field "Argumente"
%
% for futher informations go to http://ewus.de/tipp-1029.html
%---------------------------------------------------------------------------
\usepackage[nonumberlist]{glossaries}
\makeglossaries
\newglossaryentry{Tile}{name={Tile},description={Ortsangabe auf dem Spielfeld mit Row (Zeile) und Column (Spalte) beschrieben. Bsp: <r:12 c:10>}}
\newglossaryentry{FogofWar}{name={Fog of War},description={Teil der Karte, der durch die eigenen Einheiten nicht mehr sichtbar ist}}
\newglossaryentry{AI}{name={AI},description={Artificial Intelligence - K�nstliche Intelligenz}}
\newglossaryentry{Bot}{name={Bot},description={AI-Agent f�r ein Computerspiel}}
\newglossaryentry{API}{name={API},description={Application Programming Interface - Programmierschnittstelle}}
\newglossaryentry{InfluenceMap}{name={Influence Map},description={Datenstruktur, die zur Berechnung des Einflusses von Spieleinheiten auf die Spielkarte dient}}
\newglossaryentry{LazyInitialization}{name={Lazy Initialization},description={Bei der Lazy Initialization wird eine Ressource erst beim erstmaligen Gebrauch initialisiert (z.B. ein Logfile erst mit dem ersten Log-Eintrag erstellt)}}

%---------------------------------------------------------------------------

% Intro:
%---------------------------------------------------------------------------
\begin{document}                              	% Start Document
\settocdepth{subsubsection}																	% Set depth of toc
\pagenumbering{roman}														
%---------------------------------------------------------------------------

% Set up header and footer
%---------------------------------------------------------------------------
\fancyhf{}																		% clean all fields
\fancypagestyle{plain}{												% new definition of plain style
	\fancyfoot[OR,EL]{\footnotesize \thepage} 	% footer right part --> page number
	\fancyfoot[OL,ER]{\footnotesize \leftmark}	% footer left part -->	chapter
	\fancyhead[C]{															% header center part --> BFH logo
		\begin{textblock}{0}[0,0](86,9)
			\includegraphics[scale=1.0]{91_bilder/bfh_de_without_text.pdf}
		\end{textblock}
	}
}

\renewcommand{\chaptermark}[1]{\markboth{\thechapter.  #1}{}}
\renewcommand{\headrulewidth}{0pt}				% no header stripline
\renewcommand{\footrulewidth}{0pt} 				% no bottom stripline

\pagestyle{plain}
%---------------------------------------------------------------------------

\setcounter{secnumdepth}{3}
\setcounter{tocdepth}{3}
% Title Page and Abstract
%---------------------------------------------------------------------------
%
% Project documentation template
% ===========================================================================
% This is part of the document "Project documentation template".
% Authors: brd3
%

\begin{titlepage}


% Red Bar and BFH-Logo absolute placed at (87,10) on A4
% Actually not a realy satisfactory solution but working.
%---------------------------------------------------------------------------
\setlength{\unitlength}{1mm}
\begin{textblock}{210}(-10,-10)
	\begin{picture}(210,32)
		\put(0,0){\color{bfhred}\rule{240mm}{30mm}}
	\end{picture}
\end{textblock}

\begin{textblock}{0}[0,0](84,7)
	\includegraphics[scale=1.0]{bilder/bfh_de_white.pdf}
\end{textblock}

% Titel / Untertitel / Autor:
%---------------------------------------------------------------------------
\begin{flushleft}

\vspace*{4cm}

\fontsize{18pt}{20pt}\selectfont
Bachelor Thesis 20XX \\
\fontsize{12pt}{15pt}\selectfont\vspace{0.5em}
Fachbereich Elektro- und Kommunikationstechnik

\vspace{2cm}

\fontsize{30pt}{32pt}\selectfont 
\noindent \textbf{Titel Bachelor Thesis} \\

\fontsize{18pt}{20pt}\selectfont\vspace{0.3em}
Untertitel \\

\vspace{4cm}
\fontsize{12pt}{15pt}\selectfont
\begin{tabbing}
xxxxxxxxxxxxxxx\=xxxxxxxxxxxxxxxxxxxxxxx \kill
Studierende:	\> Test Peter			\\
							\> Muster Rosa		\\
																\\
Professoren:	\> Dr.~Xxxx Xxxx	\\
							\> Dr.~Yyyy Yyyy	\\
																\\
Experte:			\> Dr.~Zzzz Zzzz	\\
																\\
Datum:				\> \today					\\
\end{tabbing}
\end{flushleft}

\vspace{6mm}
\fontsize{12pt}{15pt}\selectfont
Dieses Dokument dient als Vorlage f�r die Erstellung von Berichten nach den Richtlinien der BFH. Die Vorlage ist in \LaTeX{} erstellt und unterst�tzt das automatische Erstellen von diversen Verzeichnissen, Literaturangaben, Indexierung und Glossaren. Dieser kleine Text ist eine Zusammenfassung �ber das vorliegenden Dokument mit einer L�nge von 4 bis 6 Zeilen.

\end{titlepage}

%
% ===========================================================================
% EOF
%

\cleardoubleemptypage
\setcounter{page}{1}
\chapter*{Management Summary}
\label{chap:managementSummary}

Ants AI Challenge ist ein Programmierwettbewerb, bei welchem ein Bot programmiert wird der ein Ameisenvolk steuert. Das Ameisenvolk soll auf einer Karte Futter suchen sowie gegnerische V�lker angreifen und vernichten. Dabei m�ssen Problem wie die Pfadsuche, das Verteilen von Aufgaben sowie das Schwarmverhalten gel�st werden. In unserer Arbeit wollten wir herausfinden was es alles braucht um einen solchen intelligenten Bot zu schreiben und gegen andere Mitspieler anzutreten. Wir konzentrierten uns auf die Aufgabenverteilung sowie die Pfadsuche. Diese Erfahrungen wollen wir f�r die Bachelorarbeit mitnehmen, wo wir an der n�chsten AI-Challenge, die voraussichtlich im August beginnt, aktiv teilnehmen m�chten, oder unsere Implementierung f�r diese Challenge verbessern.

\vspace{5cm}
\begin{tabbing}
xxxxxxxxxxxxxxxxxxxx\=xxxxxxxxxxxxxxxxxxxxxxx \kill
Datum					\> \today \\ \\ \\
Name Vorname	\> Lukas Kuster \\ \\
Unterschrift	\> ......................................................... \\ \\ \\
Name Vorname	\> Stefan K�ser \\ \\
Unterschrift	\> .........................................................
\end{tabbing}


%---------------------------------------------------------------------------

% Table of contents and listings
%---------------------------------------------------------------------------
\tableofcontents
\listoffigures
\listoftables
\renewcommand*{\lstlistlistingname}{Listingsverzeichnis}
\lstlistoflistings
\cleardoublepage
%---------------------------------------------------------------------------
	
% Main part:
%---------------------------------------------------------------------------
\pagenumbering{arabic}

\chapter{Einleitung}
\label{chap:einleitung}
Nachdem wir uns im Rahmen des Moduls ''Projekt 2`` (7302) mit der Implementierung eines Bots f�r den Online-Wettbewerb AI-Challenge (Ants) besch�ftigt hatten, haben wir uns f�r die Bachelorarbeit eine Verbesserung dieses Bots vorgenommen. Die AI-Challenge ist ein Wettbewerb, der im Herbst 2011 zum 3. Mal stattfand und jedes Jahr mit einem anderen Spiel durchgef�hrt wird. Ziel ist es jeweils, einen Bot zu programmieren, der durch geschickten Einsatz von KI-Technologien das Spiel m�glichst erfolgreich bestreiten kann. In dieser Durchf�hrung ging es darum, ein Ameisenvolk durch Sammeln von Ressourcen und Erobern von gegnerischen H�geln zum Sieg �ber die gegnerischen Ameisen zu f�hren.

Im ''Projekt 2`` hatten wir zwar einen Bot implementiert, der alle Aspekte des Spiels einigermassen beherrscht, also Nahrung sammeln, die Gegend entdecken, H�gel erobern und verteidigen, sowie gegen feindliche Ameisen k�mpfen. Einige dieser F�higkeiten waren aber eher rudiment�r ausgebaut, da wir uns  vor allem auf die Pfadsuche konzentriert hatten.

In der Bachelorarbeit ging es nun darum, die taktischen und strategischen Fertigkeiten des Bots auszubauen. Der Schwerpunkt lag auch bei der Bachelorarbeit nicht auf der Optimierung einer Teilaufgabe, sondern auf der Implementierung eines ausgewogenen Bots, der alle Aspekte des Spiels gleichermassen gut beherrscht.

Besonderes Augenmerk legten wir dabei auf einen modularen Aufbau des Codes. Nebst einem sauberen objektorientierten Programmdesign spiegelt sich das vor allem in den separaten Modulen ''AITools-Api``, ''Search`` und ''Strategy``, die so generisch implementiert wurden, dass sie mit geringem Aufwand auch in anderen Projekten einsetzbar sind.
\newpage

%L DONE
\chapter{Spielbeschrieb}
\label{chap:spielbeschrieb}

\section{Der Wettbewerb}
\label{sec:spielbeschrieb_wettbewerb}
Die AI Challenge\footnote{\url{http://www.aichallenge.org}} ist ein internationaler Wettbwerb des University of Waterloo Computer Science Club der im Zeitraum Herbst 2011 bis Januar 2012 zum 3. Mal stattgefunden hat. Das Spiel ist ein zugbasiertes Multiplayerspiel in welchem sich Ameisenv�lker gegenseitig bek�mpfen. Ziel einer AI-Challenge ist es, einen Bot zu schreiben, der die gegebenen Aufgaben mit m�glichst intelligenten Algorithmen l�st. Die zu l�senden Aufgaben der Ants AI Challenge sind die Futtersuche, das Explorieren der Karten, das Angreifen von gegnerischen V�lkern und deren Ameisenhaufen sowie dem Sch�tzen des eigenen Ameisenhaufen.

\section{Spielregeln}
\label{sec:spielbeschrieb_spielregeln}
Nachfolgend sind die wichtigsten Regeln, die w�hrend dem Spiel ber�cksichtigt werden m�ssen, aufgelistet.
\begin{itemize} 
\item Pro Zug k�nnen alle Ameisen um ein Feld (vertikal oder horizontal) verschoben werden.
\item Pro Zug steht insgesamt eine Rechenzeit von einer Sekunde zur Verf�gung. Es d�rfen keine Threads erstellt werden.
\item Bewegt sich eine Ameise in die 4er Nachbarschaft eines Futterpixel, wird dieses eingesammelt. Beim n�chsten Zug entsteht bei dem Ameisenh�gel eine neu Ameise.
\item Die Landkarte besteht aus passierbaren Landpixel sowie unpassierbaren Wasserstellen.
\item Ein Gegener wird geschlagen, wenn im Kampfradius der eigenen Ameise mehr eigene Ameise stehen als gegnerische Ameisen im Kampfradius der Ameise die angegriffen wird.
\item Ein Gegner ist ausgeschieden wenn alle seine eigenen Ameisenh�gel vom Gegner vernichtet wurden. Pro verlorenem H�gel gib es einen Punkteabzug. Pro feindlichen H�gel, der zerst�rt wird gibt es zwei Bonuspunkte.
\item Steht nach einer definierbaren Zeit (Anzahl Z�ge) kein Sieger fest, wird der Sieger anhand der Punkte ermittelt. 
\end{itemize}
Die ausf�hrlichen Regeln k�nnen auf der Webseite nachgelesen werden: \url{http://aichallenge.org/specification.php}

\section{Schnittstelle}
\label{sec:spielbeschrieb_schnittstelle}
Die Spielschnittstelle ist simpel gehalten. Nach jeder Spielrunde erh�lt der Bot das neue Spielfeld mittels String-InputStream, die Spielz�ge gibt der Bot dem Spielcontroller mittels String-OutputStream bekannt. Unser MyBot leitet von der Basis-Klasse Bot\footnote{Die Klasse ist im Code unter ants.bot.Bot.Java auffindbar } ab. Ein Spielzug wird im folgendem Format in den Output-Stream gelegt:
\newline
\newline
o <Zeile> <Spalte> <Richtung>
\newline
\newline
Beispiel:
\begin{verbatim}
o 4 7 W
\end{verbatim}
Die Ameise wird von der Position Zeile 4 und Spalte 7 nach Westen bewegt.
\newline
Der Spielcontroller ist in Python realisiert, der Bot kann aber in allen g�ngigen Programmiersprachen wie Java, Python, C\#, C++ etc. geschrieben werden.
 %DONE
\section{Abgrenzungen}
\label{sec:einleitung.Abgrenzungen}

Da unsere Arbeit auf dem vorg�ngigen Modul 'Projekt 2' aufbaut wurden nicht alle Module w�hrend der Bachelorarbeit erstellt. Da wir im Modul 'Projekt 2' auf einen sauberen Aufbau geachtet haben, war es uns m�glich die meisten Komponenten zu �bernehmen. Es folgt eine Auflistung was bereits bestand bzw. was wir noch erweitert haben.

\renewcommand{\arraystretch}{1.5}
\begin{table}[H]
	\centering
 \begin{tabular}{l p{10cm}}
  \textbf{Erstellt in Modul ''Projekt 2``} & \textbf{Erweiterung w�hrend der Bachelorarbeit} \\
	\hline
  Grundfunktionalit�ten des Bots & \\ 
  Pfadsuche Simple, A*, HPA*  & Auslagerung in ein eigenes Framework, Performanceverbesserungen, Erweiterung des HPA* Clustering, Pfadsuche mittels InfluenceMap\\
  Logging											& Loggen in verschiedene Logfiles \\
  Tasks 		& Die Funktionsweise jedes Tasks wurden nochmals umgekrempelt \\
  Missionen & Die Missionen wurden verfeinert und mit strategischen und taktischen Entscheidungen erweitert \\
  MinMax-Algorithmus & Bei den Kampfsituationen wurde versucht den MinMax Algorithmus einzubauen, welcher wir bereits im Modul Spieltheorie erarbeitet haben. (Mehr dazu siehe Kapitel Strategie) \\
 \end{tabular}
\caption{Abgrenzungen}
\end{table}
%S DONE, OK
\section{Projektverlauf}
\label{sec:einleitung.Projektverlauf} %L DONE
\section{Projektorganisation}
\label{sec:einleitung.Projektorganisation}

\subsection{Beteiligte Personen}
\label{sec:einleitung.Projektorganisation.BetroffenePersonen}

\textbf{Studierende:}\\
 Lukas Kuster \textit{kustl1@bfh.ch} \\
 Stefan K�ser \textit{kases1@bfh.ch}
 
\textbf{Betreuung:}\\
Dr. J�rgen Eckerle \textit{juergen.eckerle@bfh.ch}

\textbf{Experte:}\\
Dr. Federico Fl�ckiger	\textit{federico.flueckiger@bluewin.ch}


\subsection{Projektmeetings}
\label{sec:einleitung.Projektorganisation.Projektmeetings}

\begin{itemize}
\item Es fand jeweils ein Treffen mit dem Betreuer alle 1-2 Wochen statt.
\item Ein Treffen mit dem Experten fand am Anfang der Arbeit statt. Ein zweites Meeting wurde von beiden Seiten nicht f�r notwendig erachtet.
\end{itemize}

\subsection{Dokumentation}
\label{sec:einleitung.Projektorganisation.Dokumentation}

Die Dokumentation soll sich am Aufbau und Inhalt des Berichts aus dem Projekt 2 anlehnen.
\begin{itemize}
\item Das Dokument beschr�nkt sich auf das Wesentliche.
\item Verwendete AI-Techniken werden erl�utert
\item Entscheidungen und deren Grundlagen sind dokumentiert.
\item Testberichte dokumentieren die durchgef�hrten Modultests. 
\item Klassendiagramme sollen einen oberfl�chlichen Detailierungsgrad haben, so dass das Wichtigste auf den ersten Blick sichtbar ist.
\item Anleitung zum Ausf�hren eines Spiels
\end{itemize}

\subsection{Abgabe}
\label{sec:einleitung.Projektorganisation.Abgabe}
Folgende Lieferobjekte werden am Ende der Arbeit abgegeben.
\begin{itemize}
\item Dokumentation
\item Sourcecode
\end{itemize}
%L DONE
\section{Tools}
\label{sec:einleitung.Tools}

F�r diese Arbeit verwendeten wir folgende Tools:
\begin{itemize}
\item Eclipse f�r die Java-Entwicklung (\url{http://www.eclipse.org})
\item ANT f�r die Build-Automatisierung (\url{http://ant.apache.org})
\item Git (\url{http://git-scm.com}) f�r die Versionskontrolle, mit einem zentralen Repository f�r die einfachere Zusammenarbeit auf GitHub (\url{http://www.github.com})
\item TeXnicCenter (\url{http://www.texniccenter.org}) und MiKTeX (\url{http://miktex.org}) f�r die Dokumentation in \LaTeX
\item GanttProject (\url{http://www.ganttproject.biz}) f�r den Zeitplan
\item Visual Paradigm for UML (\url{http://www.visual-paradigm.com/product/vpuml/}) f�r die UML-Diagramme
\end{itemize}%L DONE
\section{Artefakte}
\label{sec:einleitung.Artefakte}

Folgende Artefakte werden zusammen mit dieser Dokumentation in elektronischer Form abgegeben:
\begin{itemize}
\item Der komplette Source-Code inklusive Git-History
\item Ein Archiv mit der generierten Code-Dokumentation (Javadoc)
\item Ein Archiv mit den Testprotokollen
\item Das Pflichtenheft
\end{itemize}
%L DONE

\newcommand{\vspacecm}{0.7cm}

\chapter{Ziele}

Der im Rahmen von Projekt 2 entwickelte Bot soll um Logik f�r taktische und strategische Entscheidungen und koordinierte Bewegung erweitert werden.

\section{Funktionale Anforderungen}
\label{sec:ziele.FunktionaleAnforderungen}

\subsection{Musskriterien}
\label{sec:ziele.FunktionaleAnforderungen.Musskriterien}

Der Bot unterscheidet zwischen diversen Aufgaben:
	\begin{itemize}
	\item Nahrungsbeschaffung
	\item Angriff
	\item Verteidigung
	\item Erkundung
	\end{itemize}

\vspace{\vspacecm}

Der Bot kann eine Beurteilung seiner Situation auf dem Spielfeld vornehmen
	\begin{itemize}
	\item Dominante/unterlegene Position
	\item Sicherheit verschiedener Gebiete des Spielfelds (eigener/gegnerischer Einfluss)
	\item Konfliktpotenzial in verschiedenen Gebieten des Spielfelds
	\end{itemize}
		
Anhand der Situationsbeurteilung werden die unterschiedlichen Aufgaben entsprechend gewichtet. Stark gewichtete Aufgaben erhalten mehr Ressourcen (Ameisen) zur Durchf�hrung.

\vspace{\vspacecm}

Der Bot identifiziert zur Erf�llung dieser Aufgaben taktische Ziele:
	\begin{itemize}
	\item Gegnerische H�gel angreifen, was bei Erfolg den Score erh�ht und das eigentliche Ziel des Spiels ist.
	\item Isolierte gegnerische Ameisen angreifen.
	\item Schwachstellen in der gegnerischen Verteidigung ausnutzen.
	\item Engp�sse im Terrain sichern bzw. versperren.
	\item Konfliktzonen, d.h. viele Ameisen auf einem engen Raum, erkennen und entsprechend reagieren.
	\end{itemize}

\vspace{\vspacecm}

F�r die konkrete Erreichung dieser definierten Ziele verf�gt der Bot �ber taktische Logik:
	\begin{itemize}
	\item Bei der Pfadsuche wird die Sicherheit der zu durchquerenden Gebiete ber�cksichtigt
	\item In Kampfsituationen kann der Bot die Ameisen in Formationen gliedern, die geeignet sind, eine lokale �berzahl eigener gegen�ber gegnerischen Ameisen zu erzeugen
	\item Beim Aufeinandertreffen mit gegnerischen Ameisen wird entschieden, ob angegriffen, die Stellung gehalten oder gefl�chtet wird.
	\end{itemize}

\subsection{Kannkriterien}
\label{sec:ziele.FunktionaleAnforderungen.Kannkriterien}
Das Verhalten des Bots ist konfigurierbar, so dass zum Beispiel ein �agressiver� Bot gegen einen defensiven Bot antreten kann.

\section{Nicht funktionale Anforderungen}
\label{sec:ziele.NichtFunktionaleAnforderungen}

\subsection{Musskriterien}
\label{sec:ziele.NichtFunktionaleAnforderungen.Musskriterien}
Modularer Aufbau f�r eine gute Testbarkeit der Komponenten.

Wichtige Funktionen wie die Pfadsuche und die Berechnung von Influence Maps sollen in separaten Modulen implementiert werden, damit sie auch von anderen Projekten verwendet werden k�nnten.

Die Codedokumentation ist vollst�ndig und dient der Verst�ndlichkeit.

\subsection{Kannkriterien}
\label{sec:ziele.NichtFunktionaleAnforderungen.Kannkriterien}
F�r die wiederverwendbaren Module wird jeweils ein kleines Tutorial geschrieben, wie die Module verwendbar sind.


\section{Abgrenzungskriterien}

Da der Wettbewerb Ants AI Challenge bereits beendet ist, ist es uns nicht m�glich dem Wettbewerb teilzunehmen.%L DONE
\section{Herausforderungen}
\label{sec:einleitung.Herausforderungen}

\subsection{Module testen}
\label{sec:einleitung.Herausforderungen.ModuleTesten}

Ein neuer Algorithmus oder eine neue Idee ist schnell mal in den Bot integriert, doch bringen die geschriebenen Zeilen den gew�nschten Erfolg? Was wenn der neue Codeabschnitt �usserst selten durchlaufen wird und dann noch fehlschl�gt? Wie wissen wir welche Ameise genau diesen n�chsten Schritt macht?

Um diese Probleme zu bew�litgen haben wir ein ausgekl�geltes Logging auf die Beine gestellt, in welchem wir schnell an die gew�nschten Informationen gelangen. (siehe \ref{sec:module.Logging})  Zudem k�nnen wir dank der Erweiterung des HTML-Viewer sofort sehen, welches die akutelle Aufgabe jeder einzelen Ameise ist. (siehe \ref{sec:module.Logging.Addon}) Weitergeholfen haben uns auch etliche Unit- und Funktionstests, mit welchen wir neu geschriebenen Code testen und auf dessen Richtigkeit pr�fen konnten. (siehe \ref{sec:testCenter.UnitundFunktionstests})

\subsection{TODO}
\label{sec:einleitung.Herausforderungen.TODO}
lorem ipsum mehr herausfoderungen ??

\subsection{Vergleich mit Bots aus dem Wettbewerb}
\label{sec:einleitung.Herausforderungen.VergleichBots}
Nach Ablauf des Wettbewerbs im Januar 2012, haben einige der Teilnehmer ihren Bot zug�nglich gemacht. Dadurch war es uns m�glich unseren Bot gegen Bots antreten zu lassen die tats�chlich am Wettbewerb teilgenommen haben. So konnten wir auch eine wage Einsch�tzung machen wir stark unser Bot ist. Mehr dazu unter siehe \ref{sec:testCenter.TestreportProfile}.%L DONE
\section{Fazit}
\label{sec:einleitung.Fazit}
TODO%L
\section{Weiterf�hrende Arbeiten}
\label{sec:einleitung.{Weiterf�hrendeArbeiten}

M�gliche weiterf�hrende Arbeiten sind...%L DONE


\chapter{Architektur}
\label{chap:Architektur}

\section{Modulabh�ngigkeiten}
\label{sec:architektur.Modulabh�ngigkeiten}

\begin{figure}[H]
\centering
\includegraphics[width=0.8\textwidth]{91_bilder/modulesOverview}
\caption{Module}
\label{fig:modulesOverview}
\end{figure}

\begin{figure}[H]
\centering
\includegraphics[width=0.8\textwidth]{91_bilder/modulesDependencies}
\caption{Modulabh�ngigkeiten}
\label{fig:modulesDependencies}
\end{figure}%L DONE

\chapter{API}
\label{sec:module.API}%S DONE OK
\chapter{Suchalgorithmen}
\label{sec:module.Suchalgorithmen}

Das erstellt Suchframework bietet eine Pfadsuche durch drei verschiedenen Algorithmen an. Zudem eine Breitensuche und eine Abwandlung davon, die Barriersuche. Als Erg�nzung wurde auch ein PathSmoothing-Algorithmus implementiert.

\section{Enities f�r die Pfadsuche}
\label{sec:module.Suchalgorithmen.Enities}

Abbildung \ref{fig:pathfinderEntities} zeigt die wichtigsten Klassen, die f�r die Pfadsuche verwendet werden.  Der \"Ubersichtlichkeit wegen wurden nur die wichtigsten Attribute und Operationen in das Diagramm aufgenommen. Die hellgr�nen Interfaces stammen von der AITools API und wurden bereits dort erl�utert.

\begin{figure}[H]
\centering
\includegraphics[width=0.9\textwidth]{91_bilder/pathfinderEntities}
\caption{Spiel-Elemente f�r die Suche (vereinfacht)}
\label{fig:pathfinderEntities}
\end{figure}

TODO PathPiece, Tile, AbstractWrapAround sind nicht in diesem?
\\TODO interface gr�n?

\begin{itemize}
\item
\textbf{Edge}: Repr�sentiert eine Kante und wird f�r das Clustering verwendet.
\item
\textbf{DirectedEdge}: Erweitert die Klasse Edge, indem die Kante in dieser Klasse gerichtet ist.
\item
\textbf{Cluster}: Der Cluster, ist ein Kartenausschnitt und wird vom HPA* Algorithmus (Kapitel \ref{subsec:module.Suchalgorithmen.Pfadsuche.HPAstar}) genutzt, indem er Pfade die Pfade kennt, die durch diesen Kartenabschnitt f�hren.
\item
\textbf{Vertex}: Ein Vertex ist ein Knoten der Teil bei Pfadsuche ist. Er verbindet weitere Konten durch Katen (Edges).
\item
\textbf{Clustering}: Das Clustering ist auch Teil des HPA* Algorithmus. Es ist f�r die Aufteilung der Karte in mehrere Clusters zust�ndig.
\end{itemize}

Die Klasse SimplePathFinder  kapselt und f�hrt verschiedene Suchstrategien aus. F�r den HPA* Algorithmus wurde diese Klasse durch ClusteringPathFinder erweitert. Diese beinhaltet zus�tzlich, wie der Name schon sagt, das Clustering.

\begin{figure}[H]
\centering
\includegraphics[width=0.4\textwidth]{91_bilder/pathfinderPathfinder}
\caption{Klassendiagramm Pfadsuche}
\label{fig:pathfinderPathfinder}
\end{figure}

\section{Pfadsuche}
\label{sec:module.Suchalgorithmen.Pfadsuche}
Wir haben drei unterschiedliche Pfadalgorithmen in unserem Code eingebaut. Via PathFinder-Klasse kann f�r die Pfadsuche der Algorithmus ausgew�hlt werden.

\begin{figure}[H]
\centering
\includegraphics[width=0.9\textwidth]{91_bilder/pathfinderSearch}
\caption{Suchstrategien}
\label{fig:pathfinderSearch}
\end{figure}

\subsection{Simple Algorithmus}
\label{subsec:module.Suchalgorithmen.Pfadsuche.Simple}

Der Simple Algorithmus versucht das Ziel zu erreichen indem zuerst die eine Achse, danach die andere Achse abl�uft. Sobald ein Hindernis in den Weg kommt, bricht der Algorithmus ab. In der Abbildung \ref{fig:SimplePath} sucht der Algorithmus zuerst den vertikal-horizontal Pfad. Da dieser Pfad wegen dem Wasserhindernis (blau) nicht ans Ziel f�hrt, wird der Pfad horizontal-vertikal gesucht. In dieser Reihenfolge wird ein Pfad gefunden. Dieser Algorithmus ist, wie der Name bereits aussagt, sehr einfach aufgebaut und kostet wenig Rechenzeit. Er ist nur f�r kurze Distanzen praktikabel, da er keinen Hindernissen ausweichen kann.

\begin{figure}[H]
\centering

\includegraphics[height=50mm]{91_bilder/simplepath}
\caption{Simple-Path Algorithmus}
\label{fig:SimplePath}
\end{figure}

Folgendes Codesnippet zeigt auf wie ein Pfad mittels Pfadsuche Simple gefunden wird. Ein SimplePathFinder wird mit der Karte initialisiert. Danach kann die Suche mit pf.search(...) gestartet werden. Als Parameter wird der Suchalgorithmus Strategy.Simple, der Startpunkt (Position der Ameise) und Endpunkt (Position des Futters), sowie die maximalen Pfadkosten (hier: 16) mitgegeben.

\lstset{language=Java, tabsize=4}
\begin{lstlisting}
SimplePathFinder pf = new SimplePathFinder(map);
List<Tile> path = pf.search(PathFinder.Strategy.Simple, ant.getTile(), food,16);
\end{lstlisting}

\subsection{A* Algorithmus}
\label{subsec:module.Suchalgorithmen.Pfadsuche.Astar}
Den A* Algorithmus haben wir nach dem Beschrieb im Buch TODO implementiert. Beim Algorithmus werden f�r jeden expandierten Knoten die gesch�tzten Kosten f(x) f�r die gesamte Pfadl�nge berechnet. f(x) besteht aus einem Teil g(x) welches die effektiven Kosten vom Startknoten zum aktuellen Knoten berechnet. Der andere Teil h(x) ist ein heuristischer Wert, der die Pfadkosten bis zum Zielknoten approximiert. Dieser Wert muss die effektiven Kosten zum Ziel immer untersch�tzen um zu gew�hren, dass der k�rzeste Pfad gefunden wird. Dies ist in unserem Spiel dadurch gegeben, dass sich die Ameisen nicht diagonal bewegen k�nnen, wir aber f�r den heuristischen Wert die Luftlinie zum Ziel verwenden. Die Pfadsuche wird immer bei dem Knoten fortgesetzt welcher die kleinsten Kosten f(x) hat. Da wir eine TileMap verwenden, definiert jede begebare Zelle ein Knoten. 

Die Abbildung \ref{fig:heuristicAstar} zeigt den effektiven Pfad (grau) vom zu expandierenden roten Knoten mit den minimalen Kosten von 10 Tiles. Die Luftlinie (blau) als heuristischer Wert hat aber nur eine L�nge von 7.6 Tiles. Damit erf�llt unsere Heuristik die Anforderungen des Algorithmus.

Eine Pfadsuche mit A* wird gleich ausgel�st wie die Suche mit dem Simple-Algorithmus, ausser dass als Parameter die Strategy AStar gew�hlt wird.

\lstset{language=Java, tabsize=4}
\begin{lstlisting}
SimplePathFinder pf = new SimplePathFinder(map);
List<Tile> path = pf.search(PathFinder.Strategy.AStar, ant.getTile(), foodTile,16);
\end{lstlisting}

\begin{figure}[H]
\centering
\includegraphics[height=50mm]{91_bilder/heuristicAstar}
\caption[A* Pfadsuche]{Heuristische Kosten (blau), Effektive Kosten (grau)}
\label{fig:heuristicAstar}
\end{figure}


\subsection{HPA* Algorthmus}
\label{subsec:module.Suchalgorithmen.Pfadsuche.HPAstar}

Eine Pfadsuche A* �ber alle Zellen der Spielkarte ist sehr teuer, da es viel Pfade gibt, die zum Teil nur eine oder wenige Tiles nebeneinander liegen. Es werden bis zum Schluss verschiedenen Pfaden nachgegangen, die sehr �hnlich sind. Abhilfe zu dieser sehr feinmaschigen Pfadsuche bietet der Hierarchical Pathfinding A* bei welchem die Karte in Regionen (Cluster) aufgeteilt wird. Von Cluster zu Cluster werden Verbindungspfade vorberechnet, welche der Algorithmus bei der Pfadsuche verwendet.

\subsubsection{Wieso wir HPA* gew�hlt haben}
Die Suche mit A* geht so lange gut, wie wir noch nicht so viele Ameisen haben und uns nur beschr�nkt auf der Karte bewegen. Sobald aber das Spiel fortgeschritten ist, wir viele Ameisen haben und uns auf der ganzen Karte bewegen, geht zu viel Zeit in der Pfadsuche verloren. Dies wollten wir verbessern und stiessen auf den HPA* Algorithmus. Dieser passt gut in unser Spiel, denn so k�nnen wir in den Anfangsphasen des Spiel, wo wir noch Zeit zur Verf�gung haben, die Pfade dank dem Clustering vorberechnen und k�nnen so sp�ter im Spiel diese Zeit sparen.\\
Die Implementation erfolgte nach Anlehnung an das Paper \cite{nohpa:IM}. Wobei sich im Gegensatz zum Paper unsere Clusters um eine Zelle �berschneiden. Zudem wird im Paper nur die Clusterart die wir sp�ter 'Corner' nennen beschrieben. Die Clusterart 'Centered' haben wir dazu programmiert.

\subsubsection{Clustering}
Das Clustering wird w�hrend dem ClusteringTask (siehe \ref{sec:module.Tasks}) ausgef�hrt, Dabei wird die Landkarte, wie bereits erw�hnt, in sogenannte Clusters unterteilt. Auf dem Bild \ref{fig.clusteredMap} wurde die Karte in 16 Clusters aufgeteilt.

\begin{figure}[H]
\centering
\includegraphics[height=50mm]{91_bilder/clusteredMap}
\caption[Clustereinteilung auf der Landkarte.]{Clustereinteilung auf der Landkarte. Clustergr�sse 4x4, Landkarte 16x16}
\label{fig.clusteredMap}
\end{figure}

Wir unterschieden zwischen den zwei Clusterarten \textit{Centered} und \textit{Corner}. Die Variante Corner wurde bereits im Vormodule 'Projekt 2' implementiert w�hrend die Variante Corner im Laufe dieser Arbeit dazu kam. Folgendes Bild zeigt den Unterschied der Varianten. Corner generiert zwei �bergangspunkte plus eine Verbindung auf der Kante zwischen zwei Clusters. Die Variante Centered generiert nur einen �bergangspunkt in der Kantenmitte der aneinander grenzenden Clusters. Die Variante Centered hat den Vorteil, dass es weniger dichtes Pfadnetz gibt, da weniger �bergangspunkte, aber sie hat auch den Nachteil, dass der gefunden Pfad, nicht in jedem Fall der K�rzeste ist. Ein Pathsmoothing muss angewendet werden.

\begin{figure}[H]
\centering
\includegraphics[height=50mm]{91_bilder/clusterArten}
\caption[Clustereinteilung auf der Landkarte.]{Vergleich der Clusterarten: Links der Typ \textit{Corner}, Rechts der Typ \textit{Centered}}
\label{fig.clusteredKinds}
\end{figure}

Nachfolgend wir erl�utert wie das Clustering vonstatten geht, verwendet wird die Custeringart Corner.\\
Nach dem Einteilen der Cluster werden f�r jeden Cluster und einen Nachbar-Cluster aus der Vierer-Nachbarschaft die Verbindungskanten berechnet. Dies kann nat�rlich nur f�r Clusters gemacht werden die auf einem sichtbaren Teil der Landkarte liegen, was zu Beginn des Spiel nicht unbedingt gegeben ist. Deshalb wird der ClusteringTask in jedem Spielzug aufgerufen, in der Hoffnung das der Cluster komplett sichtbar ist. Sobald eine beliebige Seite eines Clusters berechnet ist, wird diese Aussenkante im Cluster und dem anliegenden Nachbar gespeichert und nicht mehr neu berechnet.

\begin{figure}[H]
\centering
\includegraphics[height=50mm]{91_bilder/clusteredMap2}
\caption[Cluster mit berechneten Kanten]{Die Kanten jedes Clusters wurden berechnet}
\label{fig.clusteredMap2}
\end{figure}

Wenn ein Cluster zwei oder mehrere Aussenkanten kennt berechnet er die Innenkanten mit A*. Diese verbinden die Knoten der verschiedenen Aussenkanten. Das ergibt nun ein Pfadnetz �ber die Gesamtkarte. Im nachfolgenden Bild sind die Innenkanten (gelb) ersichtlich, die bei den ersten 8 Cluster berechnet wurden.

\begin{figure}[H]
\centering
\includegraphics[height=50mm]{91_bilder/clusteredMap3}
\caption[Cluster mit Innenkanten]{Darstellung der Innenkanten}
\label{fig.clusteredMap3}
\end{figure}

Angenommen das Clustering wurde �ber die ganze Karte vorgenommen, kann wi in der Abbildung \ref{fig.clusteredMap4} ersichtlich, ein Pfad vom Pixel (3,9) nach (13,9) mittels HPA* gesucht (gr�ne Punkte) werden. Zuerst wird eruiert in welchem Cluster sich das Start- bzw Zielpixel befindet. Danach wird in dem gefundenen Cluster ein Weg zu einem beliebigen Knoten auf der Clusterseite gesucht. Sind diese Knoten erreicht (blaue Pfade), wird nun das vorberechnete Pfadnetz mittels A* Heuristik verwendet um die beiden Knoten auf dem k�rzesten m�glichen Pfad (gelb) zu verbinden. Der resultierende Pfad k�nnte nun via Pathsmoothing noch verk�rzt werden.

\begin{figure}[H]
\centering
\includegraphics[height=50mm]{91_bilder/clusteredMap4.png}
\caption{Errechneter Weg mittels HPA*}
\label{fig.clusteredMap4}
\end{figure}

Um im Code mit einer Pfadsuche HPA* zu suchen, muss ein ClusteringPathFinder instanziert werden. Als Parameter erwartet der Konstruktor die Karte auf welcher das Clustering und die Pfadsuche gemacht wird, sowie die Clustergr�sse (hier: 10) und den Clustertyp. Das Clustering wird mit pf.update() durchgef�hrt. Danach kann die Pfadsuche gemacht werden. Falls nicht alle Clusters zur Verf�gung stehen, weil es noch unbekannte Flecke auf der Karte gibt, wird als Fallback versucht mit A* einen Pfad zu suchen.

\lstset{language=Java, tabsize=4}
\begin{lstlisting}
ClusteringPathFinder pf = new ClusteringPathFinder(map, 10, type);
pf.update();
List<Tile> path = pf.search(PathFinder.Strategy.HpaStar, start, end, -1);
\end{lstlisting}

\subsection{Pfadsuche mittels Influence Map}
\label{subsec:module.Suchalgorithmen.Pfadsuche.WithInfluenceMap}

Die Influence Map, welche wir w�hrend der Bachelorarbeit neu implementiert haben, kann auch f�r die Pfadsuche verwendet werden. Dabei sind die Pfadkosten f�r Gebiete die vom Gegner kontrolliert sind h�her als f�r neutrale Gebiete und tiefer f�r solche Gebiete die von unseren Ameisen kontrolliert werden. (Details zur Implementierung der InfluenceMap sieh Kapitel \ref{sec:module.InfluenceMap}) Die Methode getActualCost(...) in der Klasse SearchStrategy wurde erweitert. Falls die Suche mit einer InfluenceMap initialisiert wurde, sind die Kosten nicht eine Einheit pro Pfadtile, sondern k�nnen zwischen 1 (sicheres Gebiet) - 4 (gef�hrliches Gebiet) Einheiten variieren. (Die Pfadkosten d�rfen nicht negativ sein, sonst w�rde der A* Algorithmus nicht mehr korrekt funktionieren.) Die Kosten f�r jedes Pfadtile werden durch die Methode \textit{getPathCosts(...)} der InfluenceMap berechnet.

\lstset{language=Java, tabsize=4}
\begin{lstlisting}[caption={[Pfadkostenberechnung mit Ber�cksichtigung des Einflusses] Die Kosten f�r das Wegst�ck (PathPiece) werden von der Influence Map (falls verwendet) berechnet.}]
protected final int getActualCost(Node current, PathPiece piece) {
    int costOfPiece = 0;
    if (useInflunceMap)
        costOfPiece = pathFinder.getInfluenceMap().getPathCosts(piece);
    else
        costOfPiece = piece.getCost();
    return current.getActualCost() + costOfPiece;
}

\end{lstlisting}

Dadurch resultiert ein Pfad der eher durch sicheres Gebiet f�hrt. Folgende Ausgabe, welche durch einen UnitTest generiert wurde, bezeugt die korrekte Funktionalit�t. Der rote Punkt soll mit dem schwarzen Punkt durch einen Pfad verbunden werden. Auf der Karte sind zudem die eigenen, orangen Einheiten sowie die gegnerischen Einheiten (blau) abgebildet. Jede Einheit tr�gt zur Berechnung der InfluenceMap bei. Pro Tile wird die Sicherheit auf der Karte ausgegeben, negativ f�r Gebiete die vom Gegner kontrolliert werden und positiv in unserem Hoheitsgebiet.

\begin{figure}[H]
\centering
\includegraphics[width=0.99\textwidth]{91_bilder/influenceAStar01.jpg}
\caption{Ausgangslage Pfadsuche mit A* und InfluenceMap}
\label{fig.InfluenceAndPathfinding01}
\end{figure}

Ohne Ber�cksichtigung der InfluenceMap w�rde der A* Algorithmus einen Pfad finden der auf direktem Weg waagrecht zum Zielpunkt f�hrt. Sobald aber die InfluenceMap ber�cksichtigt wird, f�hrt der Pfad nicht mehr auf dem direktesten Weg zum Ziel, sondern nimmt einen Umweg �ber sicheres Gebiet. Unten abgebildet ist der k�rzeste Pfad mit Ber�cksichtigung der InfluenceMap (blau) und ohne InfluenceMap-Ber�cksichtigung (orange). 

\begin{figure}[H]
\centering
\includegraphics[width=0.99\textwidth]{91_bilder/influenceAStar02.jpg}
\caption{Resultierende Pfade mit und ohne Ber�cksichtigung der InfluenceMap}
\label{fig.InfluenceAndPathfinding02}
\end{figure}

Die Pfadkosten f�r beide Pfade verglichen, legt offen, dass je nach Ber�cksichtigung der InfluenceMap nicht der gleiche Pfad als der 'K�rzeste' von A* gefunden wird.

\renewcommand{\arraystretch}{1.5}
\begin{table}[H]
	\centering
\begin{tabular}{l | p{4cm} p{4cm}}
Pfadkosten & \textbf{ohne InfluenceMap} &\textbf{mit InfluenceMap} \\
\hline
 Oranger Pfad & \textbf{34} & 110 \\
 Blauer Pfad & 46 & \textbf{106} \\
 \end{tabular}
 \caption{Pfadkosten mit und ohne Ber�cksichtigung der InfluenceMap}
\end{table}
 
\subsection{Pathsmoothing}
\label{subsec:module.Suchalgorithmen.Pfadsuche.Pathsmoothing}

Um unseres Search-Framework zu komplettieren bietet die Pfadsuche auch ein PathSmoothing, das 'Gl�tten' eines Pfades an. Wie im Clustering schon erw�hnt, kann es sein, dass ein Pfad, der von dem HPA* Algorithmus gefunden wurde, nicht zwingend der K�rzeste ist. Die folgende Abbildung veranschaulicht wie der gefundene Pfad von Cluster zu Cluster (weisse Markierung), stets �ber die vorberechneten Verbindungspunkte (blau) verl�uft. Dies ist nicht der optimale Pfad, er kann mit PathSmoothing verk�rzt werden.

\begin{figure}[H]
\centering
\includegraphics[width=0.99\textwidth]{91_bilder/pathsmoothingOne}
\caption{Der gefundene Pfad mit HPA* Clusteringart Centered, ist nicht der k�rzeste.}
\label{fig.pathsmoothingOne}
\end{figure}

Der Algorithmus des Pathsmoothing ist (vereinfacht) wie folgt definiert. Vom Pfad der gek�rzt werden soll wird ein erster Abschnitt mit der L�nge \textit{size} genommen. Mittels manhattanDistance wird gepr�ft ob eine k�rzere Weg f�r diesen Abschnitt m�glich w�re. Falls ja wird mit A* ein neuer Pfad gesucht, sonst wird der alte Pfad (\textit{subPath}) �bernommen. Dieses Verfahren wird f�r alle nachfolgenden Pfadabschnitte gemacht, bis der ganze Pfad durchlaufen ist.\\
\\
\textbf{Rekursion:} Der beschriebene Algorithmus, hat nicht in jedem Fall den k�rzesten Pfad als Output. So sind zwar alle Pfadabschnitte optimal gek�rzt, es kann aber sein, dass wenn zwei Abschnitte zusammen gef�gt werden, der Pfad nicht mehr der k�rzeste ist. Als Beispiel: Die Wegabschnitt Z�rich-Thun und Thun-Genf m�gen optimal gek�rzt sein. Zusammengef�gt zur Strecke Z�rich-Genf, braucht es den Umweg �ber Thun nicht. Um Umweg zu entfernen wird der Algorithmus rekursiv aufgerufen, indem smoothPath(...) mit einer gr�sseren \textit{size} als Parameter f�r die zusammengesetzten Abschnitte nochmals aufgerufen wird.

\lstset{language=Java, tabsize=4}
\begin{lstlisting}[caption=JavaCode f�r das PathSmoothing]
public List<Tile> smoothPath(List<Tile> path, int size, boolean recursive) {
	int start = 0;
	int current = size;
	List<Tile> smoothedPath = new ArrayList<Tile>();
	// do while every subPath of path is checked to be shorten and added to smoothedPath
	do {
	    List<Tile> subPath = path.subList(start, current);
	    int manDist = map.manhattanDistance(subPath.get(0), subPath.get(subPath.size() - 1)) + 1;	
	    List<Tile> newSubPath = null;
	    if (manDist < subPath.size()) {
	        newSubPath = search(Strategy.AStar, subPath.get(0), subPath.get(subPath.size() - 1), subPath.size() - 1);
	    }
	    if (newSubPath != null) {
	        smoothedPath.addAll(newSubPath);
	        if (recursive && newSubPath.size() < subPath.size()) {
	            smoothedPath = smoothPath(smoothedPath, smoothedPath.size(), true);
	        }
	    } else {
	        smoothedPath.addAll(subPath);
	    }
	    start = current;
	    current = Math.min(current + size, path.size());
} while (!path.get(path.size() - 1).equals(smoothedPath.get(smoothedPath.size() - 1)));

return smoothedPath;
}
\end{lstlisting}

In der n�chsten Abbildung wurde der beschriebene PathSmoothing Algorithmus angewendet, der Pfad konnte einer urspr�nglichen Pfadl�nge   von 50 Tiles (siehe \ref{fig.pathsmoothingOne}) auf eine Pfadl�nge von 40 Tiles reduziert werden.

\begin{figure}[H]
\centering
\includegraphics[width=0.99\textwidth]{91_bilder/pathsmoothingTwo}
\caption{Der gegl�ttete Pfad, nach Anwendung des PathSmoothing-Algorithmus}
\label{fig.pathsmoothingTwo}
\end{figure}%S DONE wieso hpa*
\section{Breitensuche}
\label{subsec:module.Suchalgorithmen.Breitensuche}

Die Breitensuche (engl. breadth-first search (BFS)) war eine der Neuimplementierungen w�hrend der Bachelorarbeit. Wir verwenden diese Suche um die Umgebung einer Ameise oder eines H�gels nach Futter, Gegnern usw. zu scannen. Man k�nnte die BFS auch f�r die Pfadsuche verwenden, dies w�re aber sehr ineffizient. Im Klassendiagramm ist zu sehen auf welchen drei Methoden die Breitensuche aufbaut.

\begin{figure}[H]
\centering
\includegraphics[width=0.7\textwidth]{91_bilder/BFS}
\caption{Breitensuche Klassendiagramm}
\label{fig:BFS}
\end{figure}

TODO FRONTIER TEST fehlt auf dem Bild


Die Breitensuche wurde generisch implementierte, so dass sie vielseitig einsetzbar ist. So k�nnen zum Beispiel mittels 'GoalTest' je nach Anwendungsfall die Tiles beschrieben werden die gesucht sind. Folgende Breitensuche findet die Ameise welche am n�chsten bei einem Food-Tile <r:20,c:16> ist. Die Suche wird initialisiert indem im Konstruktor die Spielkarte mitgegeben wird, welche durchforscht wird. Zus�tzlich gilt die Einschr�nkung das die Breitensuche nur 40 Tiles durchsuchen darf, was einem Radius von zirka 7 Zellen entspricht. Falls keine Ameise gefunden wird gibt der Algorithmus NULL zur�ck.

\lstset{language=Java, tabsize=4}
\begin{lstlisting}
AntsBreadthFirstSearch bfs = new AntsBreadthFirstSearch(Ants.getWorld());
Tile food = new Tile(20,16);
Tile antClosestToFood = bfs.findSingleClosestTile(food, 40, new GoalTest() {
      @Override
      public boolean isGoal(Tile tile) {
          return isAntOnTile(tile);
      }
  });
\end{lstlisting}

Es ist auch m�glich mehrere Tiles zur�ck zu bekommen. Dazu wird die Methode \textit{findClosestTiles(...)} aufgerufen.\\
\\
Der gleiche Algorithmus kann aber auch alle passierbaren Tiles in einem gewissen Umkreis zur�ckgeben. Dies haben wir unter anderem beim Initialisieren der DefendHillMission verwendet. Wir berechnen beim Erstellen der Mission die passierbaren Zellen rundum den H�gel. Runde f�r Runde pr�fen wir diese Tiles auf gegnerische Ameisen um die entsprechenden Verteidigungsmassnahmen zu ergreifen. Der Parameter controlAreaRadius2 definiert den Radius des 'Radars' und kann je nach Profile unterschiedlich eingestellt werden.

\lstset{language=Java, tabsize=4}
\begin{lstlisting}
public DefendHillMission(Tile myhill) {
    this.hill = myhill;
    BreadthFirstSearch bfs = new BreadthFirstSearch(Ants.getWorld());
    tilesAroundHill = bfs.floodFill(myhill, controlAreaRadius2);
}
\end{lstlisting}


Um die Aufrufe der Suche im Ants-Umfeld einfacher zu gestalten haben wir die Breitensuche f�r unseren Bot mit folgenden selbst-sprechenden Methoden erweitert.

\begin{figure}[H]
\centering
\includegraphics[width=0.5\textwidth]{91_bilder/BFSants}
\caption{Breitensuche Ants-spezifisch}
\label{fig:BFSants}
\end{figure}



\subsection{Barrier (Sperre)}
\label{subsec:module.Suchalgorithmen.Breitensuche.Barrier}

Eine Erweiterung der Breitensuche erm�glicht uns eine Sperre in der Umgebung eines Ortes zu finden. Diese Verwenden wir in der DefendHillMission zum Verteidigen des eigenen H�gels. Es kann nur eine Sperre (engl. Barrier) gefunden werden wenn das Gel�nde dazu passt. Die Abbildung zeigt einen gefundene Sperre. Auf dieser H�he wird der H�gel verteidigt.

\begin{figure}[H]
\centering
\includegraphics[width=0.5\textwidth]{91_bilder/barrier}
\caption{Auf der orangen Sperre werden die Ameisen zur Verteidigung des H�gel positioniert.}
\label{fig:search.barrier}
\end{figure}

Der Algorithmus verbirgt sich in der Methode \textit{getBarrier(...)}. Diese wird mit den Parametern \textit{tileToProtect}: Ort der durch eine Sperre gesch�tzt werden soll, \textit{viewRadiusSquared}: den Sichtradius der Einheiten, den die Sperre soll weiter entfernt sein als der Sichtradius, damit die gegnerischen Einheiten nicht sehen was sich dahinter verbirgt. Der dritte Parameter \textit{maximumBarrierSize} definiert welche Breite die Sperre maximal haben darf.

\lstset{language=Java, tabsize=4}
\begin{lstlisting}
public Barrier getBarrier(final Tile tileToProtect, int viewRadiusSquared, int maximumBarrierSize) {
	int amount = BFS for getting the amount of tiles in view radius around the location to defend.
	Barrier smallestBarrier = null;
	List<Tile> tiles = get (amount + 30) tiles around the location to defend.
	
	// for loop start at the first tile not in view radius
	for(int i = amount;i<tiles.size(); i++){       
		Tile t = tiles.get(i);
		
		//vertical check
		if(!barrierVerticalInvalid.contains(t)){
			Barrier b = get vertical barrier on position of Tile t
			if(b is smaller than 5 Tiles && smaller than smallestBarrier){
				if(is barrier the only exit out of the location to defend){
						smallestBarrier = b;
				}else{
						 //add all tiles of the invaild barrier
						barrierVerticalInvalid.add(b.getTiles());
				}    					      		
			}else{
				 //add all tiles of the invaild barrier
				barrierVerticalInvalid.add(b.getTiles());
			}
		}
		
		//horizontal check
		if(!barrierHorizontalInvalid.contains(t)){
			Barrier b = get horiontal barrier on position of Tile t
			if(b is smaller than 5 Tiles && smaller than smallestBarrier){
				if(is barrier the only exit out of the location to defend){
						smallestBarrier = b;
				}else{
						 //add all tiles of the invaild barrier
						barrierHorizontalInvalid.add(b.getTiles());
				}    					      		
			}else{
				//add all tiles of the invaild barrier
				barrierHorizontalInvalid.add(b.getTiles());
			}
		}	
	}
}
\end{lstlisting}

Dank dem Abspeichern der ung�ltigen Tiles aller zu breiten Sperren in die Listen \textit{barrierHorizontalInvalid} und \textit{barrierVerticalInvalid} konnte der Algorithmus markant schneller gemacht werden. F�r diese Tiles muss nicht nochmals eine Sperre berechnet werden. Auch die if-Abfrage \textit{barrier is the only exit out of the location to defend} muss nicht mehr oft aufgerufen werden, den hinter dieser Abfrage steht n�mlich wiederum ein Test mit der Breitensuche. Dieser zus�tzliche Test mit der Breitensuche ist viel teurer als das Zwischenspeichern der Tiles aus welchen keine g�ltige Sperre gemacht werden konnte.\\
\\
Im Nachhinein hat sich ergeben, dass nicht unbedingt die schmalste Sperre die Beste w�re, sondern jede bei welche der Gegner, gel�ndebedingt, weniger Einheiten aufstellen kann. So w�re in Abbildung \ref{fig:search.barrier} eine Sperre zwei Zellen �stlicher besser, den der Gegner k�nnte beim Angriff nur mit vier Einheiten vorr�cken, die Verteidigung w�re aber mit sechs Einheiten auf einer Linie deutlich st�rker. Die Zeit hat aber hier leider nicht gereicht, den Algorithmus weiter zu verfeinern.%S DONE OK
\chapter{Strategie und Taktik}
\label{sec:module.StrategieTaktik}


\section{Influence Map}
\label{sec:module.InfluenceMap}
\begin{figure}[H]
\centering
\includegraphics[width=0.5\textwidth]{91_bilder/strategyInfluence}
\caption{Influence Map Klassendiagramm}
\label{fig.strategyInfluence}
\end{figure}

Die InfluenceMap haben wir nach den Beschreibungen in \cite{ARTIFICIALINTELLIGENCEFORGAMES} implementiert. Jede bekannte Spieleinheit auf der Spielkarte 'strahlt' eine gewissen Einfluss aus. In unserer Implementation unterscheiden wir zwischen drei Einflussradien, der Angriffsradius, der erweiterte Angriffsradius und der Sichtradius. Den Radien haben wir folgende Werte zugewiesen. 

\renewcommand{\arraystretch}{1.5}
\begin{table}[H]
	\centering
\begin{tabular}{l | r r}
 Radius & Wert & Radius in Tiles* \\
\hline
 Angriffsradius & 50 & 2.2 \\
 Erweiterter Angriffsradius & 30 & 5  \\
 Sichtradius & 10 & 8.8 \\
 \end{tabular}
\caption{Einfluss einer Spieleinheit}
\end{table}
 
 * Der Radius kann je nach Spieleinstellungen �ndern. Angegeben sind die Defaultwerte.

Wir verenden die InfluenceMap vorallem f�r die Bestimmung der Sicherheit. Abgebildet ist eine Sicherheitskarte (Desirability Map) f�r den orangen Spieler, wobei die Einflusswerte des Gegners von den Einflusswerten des eigenen Spielers je Tile subtrahiert werden. Positive Werte bedeuten sicheres Terrain und negative Werte unsicheres, vom Gegner kontrolliertes Gebiet.

\begin{figure}[H]
\centering
\includegraphics[width=0.99\textwidth]{91_bilder/influence01.jpg}
\caption{Influence Map, dargestellt ist die Sicherheit je Tile.}
\label{fig.InfluenceMap01}
\end{figure}

\subsection{Update}
\label{subsec:module.InfluenceMap.Update}

Die InfluenceMap wird zu Beginn des Spiels initialisiert, danach wird vor jeder Spielrunde ein Update gemacht. Dabei definiert ein Decay-Wert zwischen 0 und 1, wieviel von den alten Werten beibehalten wird. Folgende Formel bestimmt den neuen Wert f�r jede Zelle:


\(val_{(x,y)} = val_{(x,y)} * decay + newval_{(x,y)} * (1-decay)\)


\subsection{Anwendungsf�lle}
\label{subsec:module.InfluenceMap.Anwendungsf�lle}

In folgenden Modulen ber�cksichtigen wir Werte aus InfluenceMap um Entscheide zu f�llen.

\begin{itemize}
\item
\textbf{Pfadsuche mit InfluenceMap Ber�cksichtigung}: Siehe Kapitel \ref{subsec:module.Suchalgorithmen.Pfadsuche.WithInfluenceMap}
\item
\textbf{CombatSituation: Flucht}: M�ssen wir die Flucht ergreifen, bewegen wir unsere Ameise auf das n�chste sicherste Tile.
\item
\textbf{Abbruch Mission}: Falls eine Ameise auf einer andere Mission als die GatherFoodMission ist und ein FoodTile in seiner N�he antrifft, wird abgewogen ob die Mission zu Gunsten von Futter sammeln abgebrochen werden soll. Dabei ist ein Entscheidungsfaktor auch die 'Sicherheit' des Futters. Falls das Futter nicht auf einem sicheren Weg geholt werden kann, wird die Mission nicht abgebrochen.
\end{itemize}

Nat�rlich k�nnte man die InfluenceMap auch f�r weitere Entscheidungen verwenden. Auch der Einsatz von Spannungskarte (Tension Map), welche auch auf der InfluenceMap aufbaut, w�re denkbar. Dies wurde w�hrend dieser Arbeit nicht angeschaut bzw. implementiert.


\section{Combat Situations}
\label{sec:module.CombatSituation}

\begin{figure}[H]
\centering
\includegraphics[width=0.9\textwidth]{91_bilder/strategyTacticsCombat}
\caption{CombatPositioning Klassendiagramm}
\label{fig.strategyTacticsCombat}
\end{figure}

Kampfsituationen werden immer dann erstellt wenn gegnerische Ameisen auf unsere Ameisen treffen. Dies ist vor allem der Fall wenn ein gegnerischer H�gel angegriffen wird, oder unserer H�gel verteidigt werden muss. Eine Kampfsituation kann sich aber auch sonst wo auf der Karte ereignen.

\subsection{DefaultCombatPositioning}
\label{sec:module.CombatSituation.DefaultCombatPositioning}

DefaultCombatPositioning implementiert das Interface CombatPositioning und f�hrt die Positionierung f�r die drei Verhalten FLEE, DEFEND, ATTACK an. Das Verhalten wird in der Methode \textit{determineMode(...)} wie folgt bestimmt, wobei das 'DEFAULT' Verhalten dem ATTACK-Verhalten entspricht.

\begin{verbatim}
protected Mode determineMode() {
    final boolean enemyIsSuperior = enemyUnits.size() > myUnits.size();
    if (enemyIsSuperior)
        return Mode.FLEE;
    return Mode.DEFAULT;
}
\end{verbatim}

Wird nicht ein DefaultCombatPositioning initialisiert, sondern ein AttackingCombatPositioning (in der AttackHillMission) oder ein DefendingCombatPositioning (in der DefendHillMission) so wird das Verhalten anders bestimmt, indem die \textit{determineMode(...)}  Methode �berschrieben ist.\\
\\
\textbf{DefendingCombatPositioning}

Hier wird immer der Modus 'DEFAULT' Verhalten ausgew�hlt, aus dem Grund, dass die Verteidiger sich nicht zu weit vom H�gel entfernen in dem sie den Gegner attackieren und der H�gel von einer anderen Seite eingenommen wird. Hier k�nnte man sich Gedanken machen ob die Verteidiger, falls in �berzahl, die Gegner nicht bis zu einem gewissen Punkt angreifen sollen. Diese M�glichkeiten haben wir aber nicht genauer betrachtet.

\begin{verbatim}
protected Mode determineMode() {
    return Mode.DEFEND;
}
\end{verbatim}

\textbf{AttackingCombatPositioning}

Die Bestimmung des Modus im bei useren Angreifern in der AttackingCombatPositioning-Klasse ist da schon komplexer. Mittels Breitensuche werden die Gegner zwischen uns und dem Ziel ermittelt. Falls der Gegner TODO DISCUSS
\begin{verbatim}
  @Override
    protected Mode determineMode() {
        Tile clusterCenter = map.getClusterCenter(myUnits);
        BreadthFirstSearch bfs = new BreadthFirstSearch(map);
        int distanceToTarget = map.getSquaredDistance(clusterCenter, target);
        List<Tile> enemiesGuardingTarget = bfs.floodFill(target, distanceToTarget, new GoalTest() {
            @Override
            public boolean isGoal(Tile tile) {
                return enemyUnits.contains(tile);
            }
        });
        if ((enemiesGuardingTarget.size() * 2) <= myUnits.size())
            return Mode.ATTACK;

        // fall back to default
        return super.determineMode();
    }
\end{verbatim}

Die bereits erw�hnten Verhalten, nehmen folgende Positionierung der Ameisen vor.

\subsubsection{FLEE}
\label{sec:module.CombatSituation.DefaultCombatPositioning.Flee}

F�r jede Unit wird das sicherste Nachbarzelle mittels InfluenceMap bestimmt. Die Unit verschieb sich auf das sicherste Nachbarzelle, welche logischerweise vom Gegner entfernt liegt.

\begin{verbatim}
for (Tile myUnit : myUnits) {
    nextMoves.put(myUnit, map.getSafestNeighbour(myUnit, influenceMap));
}
\end{verbatim}

\subsubsection{DEFEND}
\label{sec:module.CombatSituation.DefaultCombatPositioning.Defend}

Der Modus DEFEND wird wird verwendet um ein Ort, bei uns unsere eigenen H�gel, zu verteidigen. Der Algorithmus definiert sich wie folgt.

\begin{verbatim}
 private void defendTarget() {
	    // if no opponents are around, just position ourselves in the diagonals
	    if (enemyUnits.isEmpty()) {
	        move enemies to the diagonals of of the defend tile
	    } else {
	    		// some sides mustn't be defend because they are surrounded by water
	        calculate sides to defend
	        	        
	        foreach(side in sides to defend) {
	        		calculate attackers on this side
	        		calculate clustercenter of  enemy
	        		calculate defenders for this side
	            calculate defend positions
	            positioning of the defenders 
	        }
	    }
    }
\end{verbatim}

Das ClusterCenter des Gegners ist jeweils der Schwerpunkt der Einheiten. Das Berechnen der Verteidiger je Seite beanspruchte im Code ein bisschen mehr als eine Zeile. Alle Ameisen die bereits in einer Richtung sind wo angegriffen wird, werden dieser Verteidigungsrichtung zugewiesen. Die restlichen Ameisen werden der Richtung zugewiesen von wo sich am meisten Angreifer n�hern. So sind nun Verteidiger und Angreifer je Seite ausgemacht und es folgt die Positionierung. Das ClusterCenter des Gegners dient als Kreismittelpunkt. Der Radius ergibt sich zwischen dem ClusterCenter der Verteidiger und dem ClusterCenter des Gegners. Auf diesem Radius werden von unserem ClusterCenter aus mittels Breitensuche so viele Tiles zur Positionierung gesucht wie Verteidiger zur Verf�gung stehen. Danach werden die Verteidiger auf den gefundenen Tiles positioniert.

\begin{figure}[H]
\centering
\includegraphics[height=40mm]{91_bilder/DefendingCombatSituation00}
\caption{DefaultCombatPositioning: Berechnung der Tiles f�r die Positionierung}
\label{fig.DefaultCombatPositioning}
\end{figure}

Abbildung \ref{fig.DefaultCombatPositioning} zeigt wie die Tiles zur Positionierung f�r die s�dliche und die n�rdliche Verteidigung definiert werden. Die schwarzen Kreise siginfieren die ClusterCenter der beiden Kontrahenten. Die gefundenen Positionierungstiles f�r die n�rdliche Verteidigung liegen auf dem grauen Kreis und werden hellorangen dargestellt. Zu sehen ist auch, dass auf dem eigenen H�gel kein Positionierung m�glich ist, hier sollen neue Ameisen schl�pfen k�nnen.\\
\\
Die Idee dieser Positionierung ist, dass so eigene Ameisen gleich weit vom Zentrum des Gegners entfernt sind. Sobald der Gegner vorr�ckt muss er gegen mehrere Ameisen im Kampf antreten. Nachteil dieser Positionierung ist, dass wenn der Gegner nicht kompakt und nicht eine gleichf�rmige Angriffsformation hat, richten wir uns nur nach dem ClusterCenter aus, die einzelnen Gruppierungen der gegnerischen Ameisen werden nicht ber�cksichtigt, was ein Nachteil bei der Verteidigung zur Folge hat.\\
Wir haben uns �berlegt, dass eine Positionierung mit Hilfe der InfluenceMap diesen Nachteil beheben w�rde. Man k�nnte die neutralen Tiles die sich zwischen den Kontrahenten befindet mittels InfluenceMap herausfinden und als Front anschauen. Nach dieser Frontlinie w�rden wir uns dann unsere Einheiten ausrichten Leider fehlte uns hier die Zeit dies nachtr�glich noch zu programmieren und auszuprobieren.

\subsubsection{ATTACK}
\label{sec:module.CombatSituation.DefaultCombatPositioning.Attack}

\begin{verbatim}
 private void attackTarget() {
 		find all enemies in target direction
 		if(no enemies around){
 			move all unit in direction of the target
 		}else{
 			 calculate clustercenter of  enemy
 			 calcuate attack positions
 			 if(more than x friendly unit are already on the formation tiles)
 			 		 calcuate formation tiles nearer to the enemy
 			 positioning of the attackers
 		}
 }
\end{verbatim}

Im ATTACK werden die Tiles zur Positionierung genau gleich berechnet wie vorhin im DEFEND-Modus. Nach der Berechnung wird geschaut wie viele unserer Angreifer schon auf diesen Tiles sind. Falls sich ein definierbarer Anteil der Ameisen bereits auf den Tiles befindet wird der Radius verk�rzt. So werden Tiles berechnet die um eine Zelle n�her beim Gegner sind. Danach folgt die Positionierung auf der Ameisen auf die berechneten Tiles.%S DONE TOREVIEW 
\chapter{Ants}
\label{sec:module.Ants}

Alle Klassen im Ants Package sind spezifisch f�r die Ants AI Challenge und machen das Ger�st unseres \gls{Bot}s aus. Nachfolgend wird deren Aufbau erl�utert.

\section{State-Klassen}
\label{sec:module.Ants.State}

\begin{figure}[H]
\centering
\includegraphics[width=0.7\textwidth]{91_bilder/State}
\caption{State-Klassen (vereinfacht)}
\label{fig:StateClasses}
\end{figure}

Abbildung \ref{fig:StateClasses} zeigt eine \"{U}bersicht �ber der Spielzustands-Klassen. F�r das Diagramm wurden lediglich die wichtigsten Methoden und Attribute ber�cksichtigt. Die State-Klassen implementieren alle das Singleton-Pattern.

\subsection{Ants}
\label{sec:module.Ants.State.Ants}
Die Ants Klasse ist die zentrale State-Klasse. Sie bietet auch einfachen Zugriff auf die anderen State-Klassen. Urspr�nglich hatten wir alle Methoden, die mit dem Zugriff auf den Spielzustand zu tun hatten, direkt in der Ants Klasse implementiert, haben aber schnell gemerkt, dass das unhandlich wird. Die Ants Klasse dient jetzt vor allem als Container f�r die anderen State-Klassen und implementiert nur noch einige Methoden, die Zustands�nderungen in verschiedenen Bereichen vornehmen.

\subsection{World}
\label{sec:module.Ants.State.World}
Die World Klasse enth�lt Informationen zur Spielwelt und erweitert die AbstractWrapAroundMap aus der AI-Tools API. Hier wird die Karte abgespeichert, in der f�r jede Zelle die aktuell bekannten Informationen festgehalten werden. Das beinhaltet die Sichtbarkeit der Zelle und was die Zelle aktuell enth�lt (Ameise, Nahrung, Wasser, ...). Ausserdem werden Listen gef�hrt, wo sich die eigenen und die bekannten gegnerischen H�gel befinden. Die Klasse bietet Methoden zur Distanzberechnung und gibt Auskunft �ber einzelne Zellen, beispielsweise ob sich Nahrung in der Umgebung einer bestimmten Zelle befindet. Zudem sind die wichtigsten Methoden des SearchableMap Interfaces (\texttt{getSuccessors...()}) hier implementiert.

\subsection{Orders}
\label{sec:module.Ants.State.Orders}
In der Orders Klasse wird �ber Befehle und Missionen der einzelnen Ameisen Buch gef�hrt. In der Liste der Befehle wird zwischengespeichert, welche Ameise welche Bewegung im aktuellen Spielzug macht. Zu Beginn des Spielzuges wird diese geleert, dann von den Tasks und Missionen mit Befehlen belegt, und am Schluss des Spielzuges werden die Befehle der Spielschnittstelle �bergeben. Die Liste der Missionen ist zug�bergreifend gef�hrt, da eine Mission �ber mehrere Spielz�ge verlaufen kann. Das zentrale Verwalten der Befehle und Missionen dient dazu, sicherzustellen, dass keine widerspr�chlichen Befehle ausgegeben werden wie: Mehrere Befehle f�r eine Ameise, gleiche Ziel-Koordinaten f�r mehrere Ameisen, eine Ameise ist mehreren Missionen zugeteilt etc..

\subsection{Population}
\label{sec:module.Ants.State.Population}
Die Population Klasse dient der Verwaltung der eigenen und der gegnerischen Ameisen-V�lker. Hier werden die Ameisen mit ihren aktuellen Aufenthaltsorten festgehalten. Wenn f�r eine Ameise ein Befehl ausgegeben wird, wird die Ameise als besch�ftigt markiert. \"{U}ber die Methode \texttt{getMyUnemployedAnts()} kann jederzeit eine Liste der Ameisen abgefragt werden, die f�r den aktuellen Zug noch keine Befehle erhalten haben und f�r neue Aufgaben zur Verf�gung stehen. Die Population Klasse f�hrt zudem Buch �ber die verf�gbaren Ressourcen pro Aufgabentyp und stellt sicher, dass kein Aufgabentyp mehr Ressourcen beansprucht, als ihm zugeteilt sind. (s. Kapitel \ref{sec:module.resourceMgmt})


\section{Spiel-Elemente (Ants-Spezifisch)}
\label{sec:module.Ants.State.Entities.Ants}

\begin{figure}[H]
\centering
\includegraphics[width=0.5\textwidth]{91_bilder/antsEntities}
\caption{Ants-spezifische Elemente der Spielwelt (vereinfacht)}
\label{fig:antsEntities}
\end{figure}

Abbildung \ref{fig:antsEntities} zeigt die wichtigsten Klassen, die die Elemente des Spiels repr�sentieren. Der \"Ubersichtlichkeit wegen wurden nur die wichtigsten Attribute und Operationen in das Diagramm aufgenommen.

\subsection{Ant}
\label{sec:implementation.Entities.Ant}
Eine Ant (Ameise) geh�rt immer zu einem Spieler; �ber die Methode \texttt{isMine()} k�nnen unsere eigenen Ameisen identifiziert werden. 
Die Ant Klasse implementiert das Interface Unit aus der AITools-API, welches eine Abstraktion bietet, die die Verwendung der Ameisen in den generischen Modulen erlaubt.
Eine Ameise weiss jeweils, in welcher Zelle sie steht. Das Feld \texttt{nextTile} dient der Verfolgung einer Ameise �ber mehrere Z�ge. Der Wert des Feld wird jeweils gesetzt, wenn der Ameise ein n�chster Zug zugewiesen wird. Im n�chsten Spielzug wird die Position der Ameise durch diese Information aktualisiert.

\subsection{Route}
\label{sec:implementation.Entities.Route}
Eine Route repr�sentiert eine einfache Verbindung zwischen zwei \gls{Tile}s. Die Distanz wird zwischen den \gls{Tile}s wird mit der Luftliniendistanz gemessen.

\subsection{Ilk}
\label{sec:implementation.Entities.Ilk}
Ilk ist der Typ einer Zelle. Der Ilk einer \gls{Tile}-Instanz gibt an, was sich gerade in der Zelle befindet. Dies kann ein Gel�ndetyp sein, wenn die Zelle nicht besetzt ist, oder es kann eine Ameise, Nahrung, oder ein H�gel sein; in diesem Fall ist der Gel�ndetyp implizit ''Land``, da Wasser-\gls{Tile}s nicht besetzt sein k�nnen. Die Ilk-Enumeration bietet Hilfsmethoden, um festzustellen, ob eine Zelle passierbar oder besetzt ist.
%S DONE TOREVIEW
\section{Aufbau Bot}
\label{sec:module.AufbauBot}


\subsection{Klasse Bot}
\label{sec:module.Bot.Bot}

\begin{figure}[H]
\centering
\includegraphics[width=0.7\textwidth]{91_bilder/antsBot}
\caption{Vererbung der Bots wobei auf Stufe impl (Implementation) nur MyBot verwendet wird.}
\label{fig:antsBot}
\end{figure}

Als Basis f�r unsere Bot Implementation haben wir den Beispiel-Bot (Klasse Bot.java) verwendet, der im Java-Starter-Package enthalten ist, das von der AI-Challenge-Website heruntergeladen werden kann. Dieser erbt von den Klassen AbstractSystemInputReader und AbstractSystemInputParser, die die Interaktion mit der Spiele-Engine �ber die System-Input/Output Streams kapseln. F�r eine optimierte L�sung k�nnte der Bot auch angepasst werden, indem er selber auf die Streams zugreift. Im Rahmen dieser Arbeit erschien uns das aber noch nicht n�tig. Die Klasse Bot.java dient als Grundlage f�r die Klasse BaseBot, welcher wiederum Grundlage ist f�r die Finale Klasse MyBot.java.



\subsection{BaseBot}
\label{sec:module.Bot.BaseBot}

Die abstrakte Klasse BaseBot erbt vom Bot. Hier haben wird die Struktur unseres Spielzuges definiert.

\subsection{Ablauf eines Zugs} 
\label{sec:implementation.Bot.Turn}

\begin{figure}[H]
\centering
\includegraphics[width=0.9\textwidth]{91_bilder/FirstTurn}
\caption{Ablauf des ersten Zugs des Spiels}
\label{fig:firstTurn}
\end{figure}

\begin{figure}[H]
\centering
\includegraphics[width=0.9\textwidth]{91_bilder/Turn}
\caption{Ablauf der weiteren Z�ge des Spiels}
\label{fig:turn}
\end{figure}


TODO calculateInfluence() fehlt 

Abbildung \ref{fig:firstTurn} zeigt den Ablauf des ersten Zugs, w�hrend Abbildung \ref{fig:turn} den Ablauf aller weiteren Z�ge zeigt. 

Jeder Zug beginnt mit dem Einlesen des Inputs vom SystemInputStream. Wenn der Bot das Signal ''READY`` (1. Zug) oder ''GO`` (alle weiteren Z�ge) erh�lt, kann er den gesammelten Input verarbeiten (Methode parseSetup() resp. parseUpdate()). Danach wird die eigentliche Logik des Bots in der Methode doTurn(...) ausgef�hrt.

Im 1. Zug werden dabei Instanzen der Tasks erstellt. Abgesehen davon unterscheidet sich der 1. Zug von diesem Punkt an nicht mehr von allen nachfolgenden Z�gen. Die Tasks werden vorbereitet. (Aufruf der jeweiligen setup() Methode. Danach werden einige statistische Werte aktualisiert und in jedem 10. Zug auch geloggt. Dann werden die Tasks in der definierten Reihenfolge aufgerufen. Hier wird der L�wenanteil der Zeit verbracht, denn die Tasks enthalten die eigentliche Logik unserer Ameisen.

Zum Schluss werden dann mit issueOrders() die Z�ge der Ameisen �ber den SystemOutputStream an die Spielengine �bergeben. Im Code sieht das ganze folgendermassen aus.

\begin{verbatim}
@Override
/*
 * This is the main loop of the bot. All the actual work is
 * done in the tasks that are executed in the order they are defined.
 */
public void doTurn() {
		// write current turn number, ants amount into the log file.
	addTurnSummaryToLogfiles();
		// new calculation of the influence map
	calculateInfluence();
		// write some statistics about our population
	doStatistics();
		// initialize the task (abstract method) must be implemented by the inherited class
	initTasks();
		// execute all task (main work to do here)
	executeTask();
		// write all orders to the output stream
	Ants.getOrders().issueOrders();
		// log all ants which didn't get a job.
	logUnemployedAnts();
}
\end{verbatim}


\subsection{MyBot}
\label{sec:module.Bot.MyBot}

Wie bereits erw�hnt ist die Methode initTasks() in BaseBot abkstrakt und muss von MyBot implementiert werden. InitTasks() definiert welche Tasks, oder besser gesagt F�higkeiten der Bot hat. Dies wurde ausgelagert, da nicht nur MyBot von BaseBot erbt, sondern auch weiter Bots die wir zu Testzwecken erstellt haben um nur gewisse Funktionalit�ten zu testen. (Siehe dazu \ref{sec:testCenter.Testbots}) Weiter wird in MyBot initLogging(...) aufgerufen. Hier definieren wir welche Logkategorien mit welchem Loglevel ins Logfile geschrieben werden. Mehr zum Thema Logging ist im Kapitel Logging zu finden. Je nach Modul das gerade getestet wird k�nnen die Anzahllogeintr�ge justiert werden.\newline
MyBot initialisiert folgende Tasks; es sind die Tasks die sich w�hrend der Arbeit bew�hrt haben. Die detailierte Beschreibung der Tasks ist im Kapitel \ref{sec:module.Tasks} zu finden.

\begin{itemize}
		\item GatherFoodTask
		\item AttackHillsTask
		\item DefendHillTask
		\item ExploreTask
		\item ClearHillTask
		\item CombatTask
		\item ClusteringTask
\end{itemize}%S DONE TOREVIEW
\section{Tasks}
\label{sec:module.Tasks}

\begin{figure}[H]
\centering
\includegraphics[width=0.5\textwidth]{91_bilder/Tasks}
\caption{Tasks}
\label{fig:tasks}
\end{figure}
Zu Beginn des Projekts haben wir die wichtigsten Aufgaben einer Ameise identifiziert. Diese Aufgaben wurden als Tasks in eigenen Klassen implementiert. Das Interface Task\footnote{Das Interface ist im Code unter ants.tasks.Bot.Java auffindbar.} definiert eine setup()-Methode welche den Task initiiert, sowie eine perform()-Methode welche den Task ausf�hrt. Im Programm werden die Tasks nach deren Wichtigkeit ausgef�hrt, was auch der nachfolgenden Reihenfolge entspricht. Jedem Task stehen nur die unbesch�ftigten Ameisen zur Verf�gung, d.h. jene welchen noch keine Aufgabe zugeteilt wurde.

\subsection{MissionTask}
\label{subsec:implementation.Tasks.MissionTask}
Dieser Task pr�ft alle aktuellen Missionen auf deren G�ltigkeit, beispielsweise ob die Ameise der Mission den letzten Zug �berlebt hat und die Mission weiterf�hren kann. Falls g�ltig, wird der n�chste Schritt der Mission ausgef�hrt.

\subsection{GatherFoodTask}
\label{subsec:implementation.Tasks.GatherFoodTask}
F�r jedes Food-Tile werden in einem definierbaren Radius r die n�chsten Ameisen bestimmt. Danach wird nach aufsteigender Luftliniendistanz mit dem Pfadsuchalgorithmus SIMPLE (s. Abschnitt \ref{subsec:implementation.Pfadsuche.Simple}) oder -- falls dieser keinen Pfad gefunden hat -- mit A* eine passierbare Route gesucht. Wenn ein Pfad existiert, kann mit der Ameise und dem Food-Tile eine GatherFoodMission erstellt werden, welche die Ameise zum Food-Tile f�hrt. Zu jedem Food-Tile wird immer nur eine Ameise geschickt.

\subsection{AttackHillsTask}
\label{subsec:implementation.Tasks.AttackHillsTask}
Sobald gegnerische Ameisenhaufen sichtbar sind, sollen diese angegriffen werden. Das Zerst�ren eines gegnerischen Haufens ist wie erw�hnt 2 Punkte wert. Die Kriterien, nach denen eine Pfad zum gegnerischen Haufen gesucht wird, sind die selben wie beim GatherFoodTask, ausser dass mehrere Ameisen das Ziel angreifen k�nnen. Es wird eine AttackHillMission erstellt.

\subsection{CombatTask}
\label{subsec:implementation.Tasks.CombatTask}
Beim Angriffstask wird berechnet ob wir in einem Kampfgebiet (definiert �ber den Sichtradius einer Ameise) die �berhand, d.h. mehr Ameisen platziert haben. Falls ja, wird die gegnerische Ameise angegriffen.

\subsection{ExploreTask}
\label{subsec:implementation.Tasks.ExploreTask}
F�r alle noch unbesch�ftigten Ameisen wird mittels ManhattanDistance der n�chste Ort gesucht, der noch nicht sichtbar, also unerforscht ist. Falls ein Pfad mittels Pfadsuchalgorithmus gefunden wird, wird eine ExploreMission (s. Abschnitt \ref{sec:implementation.Missionen}) erstellt. Die Ameise wird den gefundenen Pfad in den n�chsten Spielz�gen ablaufen.

\subsection{FollowTask}
\label{subsec:implementation.Tasks.FollowTask}
Der FollowTask ist f�r Ameisen angedacht welche aktuell keine Aufgabe haben. Diese Ameisen sollen einer nahe gelegenen, besch�ftigten Ameise folgen, damit diese nicht alleine unterwegs ist.

\subsection{ClearHillTask}
\label{subsec:implementation.Tasks.ClearHillTask}
Dieser Task bewegt alle Ameisen, welche neu aus unserem H�gel ''schl�pfen``, vom H�gel weg. So werden nachfolgende Ameisen nicht durch diese blockiert.

\subsection{ClusteringTask}
\label{subsec:implementation.Tasks.ClusteringTask}
Der ClusteringTask wird als Vorbereitung f�r den HPA* Algorithmus verwendet. Hier wird f�r alle sichtbaren Kartenregionen ein Clustering vorgenommen. Das Clustering wird im Kapitel \ref{subsec:implementation.Pfadsuche.HPAstar} im Detail beschreiben.
%S DONE TOREVIEW TODO: CombatTask
\section{Missionen}
\label{sec:module.Missionen}
\begin{figure}[H]
\centering
\includegraphics[width=0.9\textwidth]{91_bilder/Missions}
\caption{Missionen}
\label{fig:missions}
\end{figure}
Eine Mission dauert �ber mehrere Spielz�ge. Die meisten Missionen (GatherFoodMission, ExploreMission, AttackHillMission, AttackAntMission) sind Pfadmissionen\footnote{Die abstrakte Klasse PathMission ist im Code unter ants.missions.PathMission.java auffindbar.}, bei welchen die Ameise einem vorgegebenen Pfad, der bereits beim Erstellen der Mission berechnet wurde, folgt. 
Die FollowMission ist eine spezielle Mission, mit der eine Ameise einfach einer anderen Ameise hinterherl�uft.

Eine Mission kann auch abgebrochen werden, wenn es keinen Sinn mehr macht, sie weiter zu verfolgen. Je nach spezifischer Mission sind aber die Abbruchbedingungen anders. Zum Beispiel die GatherFoodMission ist nur solange g�ltig wie das Futter noch nicht von einer anderen Ameise eingesammelt wurde.
Abbildung \ref{fig:missions} zeigt einen \"Uberblick �ber die wichtigsten Missionen und ihre Hierarchie.
%S DONE TOREVIEW
\section{Ressourcen Management}
\label{sec:module.resourceMgmt}
\subsection{Profile}
\label{sec:module.Profile}%L DONE
\chapter{Logging}
\label{sec:module.Logging}
\begin{figure}[H]
\centering
\includegraphics[width=0.8\textwidth]{91_bilder/Logging}
\caption{Logging Klassen}
\label{fig:Logging}
\end{figure}
\begin{figure}[H]
\centering
\includegraphics[width=0.5\textwidth]{91_bilder/LoggingConfig}
\caption{Logging Konfiguration}
\label{fig:LoggingConfig}
\end{figure}
Nach einem absolvierten Spiel analysierten wir jeweils die Spielsituationen, welche sich ergeben haben. Dazu geh�rte das Analysieren des geschriebenen Logs. Dabei bedienten wir uns den nachfolgenden Mechanismen.

\section{Logkategorien und Loglevel}
\label{sec:module.Logging.LogkategorienundLoglevel}

Jeder Logeintrag geh�rt einer Logkategorie an. Je Logkategorie kann der Loglevel defniert werden. Die Loglevel lauten TRACE, DEBUG, INFO und ERROR. Wenn also zum Beispiel bei der Logkategorie ATTACKHILLMISSION der Loglevel auf INFO gestellt ist, werden nur die Fehler auf Stufe INFO und ERROR in das Logfile geschieben. Zudem kann, falls erw�nscht, jede Logkategorie in ein eigenes Logfile geschrieben werden. Die meisten Module habe ihre eigene Logkategorie, so kann durch korrekte Logeinstellung erzwungen werden dass nur die Logs, welche f�r das Analysieren eines bestimmten Spielmoduls von Bedeutung sind, ins Logfile geschrieben werden. Dadurch m�ssen nicht riesige Mengen an Logs druchw�lzt werden um an die Informationen heran zu kommen.


\section{JavaScript Addon f�r HMTL-Gameviewer}
\label{sec:module.Logging.Addon}
TODO Der Aufruf ist jetzt LiveInfo.live...
Das Codepaket welches von den Challenge-Organisatoren mitgeliefert wird, bietet bereits eine hilfreiche 2D-Visualisierung des Spiels, mit welchem das Spielgeschehen mitverfolgt werden kann. Die Visualisierung wurde mit HMTL und Javascript implementiert. Leider ist es nicht m�glich zus�tzliche Informationen auf die Seite zu projizieren. Deshalb haben wir den Viewer bereits im Projekt 2 mit einer solchen Funktion erweitert. Mit der Codezeile Logger.liveInfo(...) kann eine Zusatzinformation geschrieben werden, welche auf dem Viewer sp�ter sichtbar ist. Es muss definiert werden mit welchem Zug und wo auf dem Spielfeld die Infomation angezeigt werden soll. Im Beispiel wird an der Position der Ameise (ant.getTile()) ausgegeben welchen Task die Ameise hat.
\begin{verbatim}
Logger.liveInfo(Ants.getAnts().getTurn(), ant.getTile(), 
                "Task: %s ant: %s", issuer, ant.getTile());
\end{verbatim}
Auf der Karte wird ein einfaches aber praktisches Popup mit den geschriebenen Informationen angezeigt. Dank solcher Zusatzinformationen muss nicht m�hsam im Log nachgeschaut werden, welcher Ameise wann und wo welcher Task zugeordnet ist.

\begin{figure}[H]
\centering
\includegraphics[height=70mm]{91_bilder/javascriptAddon.png}
\label{fig.javascriptAddon}
\caption[Live-Info Popupfenster]{Im Popupfenster steht die Aufgabe der Ameise sowie die Pixel des Pfades (falls vorhanden), welcher die Ameise ablaufen wird.}
\end{figure}

Das angezeigte Popup zeigt welchen Task (GatherFoodTask) die Ameise hat, wo sie sich befindet <r:28 c:14>, welches Futterpixel angesteuert wird <r:35 c:13> und welchen Pfad dazu berechnet wurde. Im Rahmen der Bachelorarbeit wurde dieses Addon erweitert. Nun werden alle Pixel welche in dem Popup ausgegeben werden auf der Karte markiert. Siehe (Abb. \ref{fig.javascriptAddon2})

\begin{figure}[H]
\centering
\includegraphics[height=45mm]{91_bilder/javascriptAddon2.jpg}
\label{fig.javascriptAddon2}
\caption[Erweiterung des Live-Info Popupfenster]{Mit der erweiterten Version wird der Pfad (orange) der Ameise von <r:48 c:21> nach <r:47 c:16> auf der Karte abgebildet.}
\end{figure}%L DONE

\chapter{TestCenter}
\label{chap:testCenter} 

todo besserer name

\section{Unit- und Funktionstests}
\label{sec:testCenter.UnitundFunktionstests}%S DONE TOREVIEW
\section{Testbots}
\label{sec:testCenter.Testbots}

Eine weitere Method war Testbots zu erstellen. Zum Beispiel haben wir den DefendHillBot erstellt der nur verteidigt nicht aber angreift. Wir nahmen eine kleine Karte, so dass es schnell zu Angriffen des Gegners kam. So konnten wir unser Verhalten in der Verteidigung nach kurzer Spieldauer genau analysieren und verbessern.\\
\\
Das selbe galt f�r den AttackHillBot. Wir haben wiederum eine kleine Karte genommen auf der viel Futter vorhand war, so dass sich das Ameisenvolk schnell vermehren konnte. Dank der beschr�nkten Karte kam es schnell zu Angriffen, wir konnten unsere Angriffspositionierung testen.\\
\\
Im Kapitel Task wurde beschrieben, welche Task bzw. Missionen nicht erfolgsversprechend waren und wir nicht weiterverfolgt haben. Bevor wir das  aber wussten, haben wir, um die Funktionen zu testen, einen speziellen Bot erstellt. Anhand der Ergebnissen konnten wir herausfinden, dass diese Methoden nicht praktikabel waren und wir die Ideen verworfen haben. So sind im Code noch folgende Bots zu finden:
\begin{itemize}
	\item ConcentrateBot
	\item FlockBot
	\item SwarmBot		
\end{itemize}




%S DONE TOREVIEW
\section{Performance Suchalgorithmen}
\label{sec:testCenter.PerformanceSuchalgorithmen}

Die entwickelten Suchstrategien werden in diesem Kapitel verglichen. Zum Vergleich wird eine Testkarte der Gr�sse 100 x 110 Tiles verwendet, auf dieser werden die Pfadsuchalgorithmen angewendet. Die Suchstrategie SIMPLE wird nicht in den Vergleich einbezogen, diese ist wie beschreiben nur f�r kurze Pfade einsetzbar. Daf�r wird der HPA* mit zwei verschiedenen Clustergr�ssen 14 \& 20, sowie den unterschiedlichen Typen Corner und Centered angewendet. Der erste Test beinhaltet zwei Durchg�nge bei welchen jeweils ein anderer Start- und Zielort definiert wurde.

\begin{table}[H]
	%weird hack to enable footnotes in the table
	\begin{minipage}{20cm}

		\begin{tabular}{ l | r  r  | r  r | r r }
		
		&  \multicolumn{2}{l|}{\textbf{1. Durchlauf}} & \multicolumn{2}{l|}{\textbf{2. Durchlauf}} &
		\multicolumn{2}{l}{\textbf{Durchschnitt}} \\
		
		\textbf{Suchalgorithmus} & \textbf{Dauer\footnote{Dauer in Millisekunden}} & \textbf{Pfadl�nge\footnote{Pfadl�nge in Anzahl Tiles}} & \textbf{Dauer} &
		\textbf{Pfadl�nge} & \textbf{Dauer} & \textbf{Pfadl�nge}  \\
		\hline
		
		A* & 124.6 & 95 & 101.2 & 95 & 112.9 &	95 \\
		HPA* (Centered,14)\footnote{Typ: Centered, Clustergr�sse: 14} & 167.6 & 99 & 25.2 & 107 & 96.4 & 103\\
		HPA* (Corner,14) & 2677.4 & 117 & 2394.6 & 117 & 2536 & 117 \\
		HPA* (Centered,20)  & 6.4 & 106 & 60 & 137 & 33.2 & 121.5 \\
		HPA* (Corner,20) & 22 & 113 & 231 & 133 & 126.5 & 123 \\
		\end{tabular}\par
		\vspace{-0.75\skip\footins}
   \renewcommand{\footnoterule}{}
  \end{minipage}
	\caption{Testresultate: Vergleich der Suchalgorithmen}
	\label{tab:testSearchAlgorithms}
\end{table}



In der Tabelle \ref{tab:testSearchAlgorithms} sind die Testresultate aufgelistet. Zu sehen ist, dass nur A* den optimalsten Pfad von 95 Tiles findet in ca. 120 Millisekunden findet. HPA* mit Typ Corner und mit der kleinerer Clustergr�sse von 14 Tiles ist langsamer als A*. Dies ist darauf zur�ckzuf�hren, dass mit dieser Einstellung das Pfadnetz von HPA* zu feinmaschig ist. Um den Pfad zu finden m�ssen viele Kanten und Knoten der geclusterten Karte durchforscht werden. Das dauert l�nger als bei A*, wo die Nachfolgerknoten einfach aus einem zweidimensionalen Array gelesen werden. Der Typ Centered ist deutlich eher schneller als der Typ Corner, da er in etwa nur halb so viele Pfadknoten. (siehe Kapitel \ref{subsec:module.Suchalgorithmen.Pfadsuche.HPAstar}) Sobald die Clustergr�sse auf 20 Tiles gestellt wird, ist HPA* deutlich schneller als A*. Allgemein muss aber gesagt werden, dass es sehr darauf an kommt ob die Gel�ndestruktur dem Clustering entgegenkommt, so dass die Verbindungspunkte g�nstig gelegen sind und einen Pfad gefunden werden kann der nicht zu viele Umwege macht. Die folgende Abbildung zeigt wie die verschiedene Suchalgorithmen den Pfad zwischen den zwei schwarzen Punkten gefunden haben.

\begin{figure}[H]
\centering
\includegraphics[height=100mm]{91_bilder/searchcompare}
\caption[A* Pfadsuche]{Vergleich der gefundenen Pfade: Viele Wege f�hren nach Rom.}
\label{fig:comapreSearchStrategies}
\end{figure}

Allerdings wurde beim ersten Tests der Pfad nicht gesmoothed, auch die Zeit, welche es brauchte um die Karte in Cluster aufzuteilen, ist nicht ber�cksichtigt. Nachfolgender Test ber�cksichtigt nun diese zwei Aspekte.

\begin{table}[H]
	%weird hack to enable footnotes in the table
	\begin{minipage}{20cm}

		\begin{tabular}{ l | r r r r r }
		
		\textbf{Suchalgorithmus} & \textbf{Clustering} & \textbf{Pfadsuche} & \textbf{PathSmoothing} & 
		\textbf{Pfadl�nge 1}\footnote{ohne Smoothing} & \textbf{Pfadl�nge 2}\footnote{mit Smoothing}  \\
		\hline		
		A* & - & 95 ms & - & 95 & -  \\
		HPA* (Centered,20)  & 484 ms & 4 ms & 16 ms & 127 & 119 \\
		\end{tabular}\par
		\vspace{-0.75\skip\footins}
   \renewcommand{\footnoterule}{}
  \end{minipage}
	\caption{Testresultate: Dauer, mit Ber�cksichtigung von Clustering und PathSmoothing}
	\label{tab:totalDuration}
\end{table}

Um die ganze Karte zu Clustern braucht es einmalig Total 484 Millisekunden. W�hrend dem Spiel verteilt sich der Aufwand f�r das Clustering auf mehrere Z�ge und ist deswegen nicht kritisch. Wenn wir einen gegl�tteten HPA* wollen brauchen wir ca. 20 ms pro Suche, f�r den optimalen Pfad mit A* 95 ms. Wenn wir also HPA* verwenden, sparen rund 70 ms pro Pfadsuche. Wenn man ber�cksichtigt, dass pro Zug f�r mehrere Ameisen ein Pfad gesucht wird, lohnt sich der Einsatz von HPA* allemal.%S => VERGLEICH EV. in TESTCENTER, HPA* corner, centered, pathsmoothing
\section{Testreport Profile}
\label{sec:testCenter.TestreportProfile}
 
Um die verschiedenen Profile unseres Bots zu testen, f�hrten wir diverse Testl�ufe durch, in denen wir die verschieden konfigurierten Bots jeweils 100 Mal gegen verschiedene Gegner und gegeneinander antreten liessen. F�r diese Testl�ufe wurden die in Tabelle \ref{tab:definierteProfile} aufgef�hrten Profile verwendet.

W�hrend der Entwicklung f�hrten wir die meisten Tests gegen den Bot von Evan Greavette (Username egreavette) durch. Er hatte seinen Bot �ber GitHub ver�ffentlicht: \url{https://github.com/egreavette/Ants-AI}. Daher f�hrten wir auch die ersten Testl�ufe gegen diesen Gegner durch:

\renewcommand{\arraystretch}{1.5}
\begin{table}[H]
	\centering
		\begin{tabular}{ l | r  r  r  r }
			\textbf{Profil} & \textbf{Siege} & \textbf{Unentschieden} & \textbf{Niederlagen} & \textbf{Total Punkte} \\
			\hline
			Default & 62 & 13 & 25 & 957:579\\
			Aggressive & 61 & 18 & 21 & 1005:642\\
			Defensive & 53 & 15 & 32 & 915:726\\
			Expansive & 59 & 16 & 25 & 955:673
		\end{tabular}
	\caption{Bilanz der Testl�ufe gegen egreavette}
	\label{tab:testAgainstEgreavette}
\end{table}

Wie man unschwer erkennen kann, sind gegen diesen Gegner drei der Profile ungef�hr gleichwertig; lediglich der Defensive Bot f�llt etwas ab.

Als n�chstes f�hrten wir einen Testlauf gegen den Sieger des Wettbewerbs durch. Mathis Lichtenberger (Username xathis) hatte seinen Bot ebenfalls �ber GitHub zur Verf�gung gestellt: \url{https://github.com/xathis/AI-Challenge-2011-bot}. 

\begin{table}[H]
	\centering
		\begin{tabular}{ l | r  r  r  r }
			\textbf{Profil} & \textbf{Siege} & \textbf{Unentschieden} & \textbf{Niederlagen} & \textbf{Total Punkte} \\
			\hline
			Default & 17 & 7 & 76 & 521:1094\\
			Aggressive & 23 & 11 & 66 & 593:1010\\
			Defensive & 11 & 8 & 81 & 479:1106\\
			Expansive & 24 & 6 & 70 & 610:988
		\end{tabular}
	\caption{Bilanz der Testl�ufe gegen xathis}
	\label{tab:testAgainstXathis}
\end{table}

Gegen diesen starken Gegner zeigen sich bereits deutlichere Unterschiede zwischen den Profilen. Man sieht, dass die beiden offensiver ausgerichteten Profile (Aggressive und Expansive) sich deutlich besser schlagen, w�hrend das Defensive Profil deutlich abf�llt. Dies entspricht in etwa unseren Erwartungen. Die St�rken von xathis' Bot liegen eindeutig in der Offensive, w�hrend die Defensive nicht das Prunkst�ck unseres Bots ist. Deshalb k�nnen wir in der Verteidigung gegen xathis fast nur verlieren und haben eine gr�ssere Chance, wenn wir selber in die Offensive gehen.

Als n�chstes liessen wir unsere 4 Profile gegeneinander antreten:

\begin{table}[H]
	\centering
		\begin{tabular}{ l | r  r  r  r }
			\textbf{Profil} & \textbf{Siege} & \textbf{Unentschieden} & \textbf{Niederlagen} & \textbf{Total Punkte} \\
			\hline
			Default & 22 & 7 & 71 & 332:891\\
			Aggressive & 23 & 10 & 67 & 314:909\\
			Defensive & 23 & 8 & 69 & 300:923\\
			Expansive & 16 & 7 & 77 & 277:946
		\end{tabular}
	\caption{Bilanz der Testl�ufe gegeneinander}
	\label{tab:test4Profile}
\end{table}

Hier zeigt sich, dass die Profile im Kampf gegeneinander relativ ausgeglichen sind. Es zeigt sich auch, dass das expansive Verhalten sich gegen mehrere Gegner weniger lohnt, da sich dabei die Ameisen eher auf dem Spielfeld verzetteln und man Gefahr l�uft, isolierte Ameisen zu verlieren.

Alles in allem waren aber diese Standard-Profile noch mehr oder weniger gleichwertig. Wir erstellten daher neue, st�rker vom Standard abweichende Profile und f�hrten noch einmal Testl�ufe gegen xathis und gegeineinander durch.

\renewcommand{\arraystretch}{1.5}
\begin{table}[H]
	\centering
		\begin{tabular}{ l | r  r  r  r }
			\textbf{Parameter} & \textbf{Default} & \textbf{Aggressive 2} & \textbf{Defensive 2} & \textbf{Expansive 2}\\
			\hline
			defaultAllocation.gatherFood & 25 & 15 & 25 & 35\\
			defaultAllocation.explore & 25 & 15 & 25 & 35\\
			defaultAllocation.attackHills & 25 & 60 & 10 & 20\\
			defaultAllocation.defendHills & 25 & 10 & 40 & 10\\
			defendHills.startTurn & 30 & 30 & 20 & 50\\
			defendHills.minControlRadius & 8 & 8 & 20 & 8\\
			explore.forceThresholdPercent & 80 & 80 & 80 & 90\\
			explore.forceGain & 0.25 & 0.1 & 0.25 & 0.4\\
			explore.dominantPositionBoost & 5 & 3 & 5 & 10\\
			attackHills.dominantPositionBoost & 2 & 10 & 2 & 2\\
			attackHills.halfTimeBoost & 20 & 40 & 15 & 20\\
		\end{tabular}
		\caption{Die Profile f�r den 2. Testlauf}
		\label{tab:definierteProfile2}
\end{table}

\begin{table}[H]
	\centering
	%weird hack to enable footnotes in the table
	\begin{minipage}{11cm}
    \centering
		\begin{tabular}{ l | r  r  r  r }
			\textbf{Profil} & \textbf{Siege} & \textbf{Unentschieden} & \textbf{Niederlagen} & \textbf{Total Punkte} \\
			\hline
			Default & 17 & 7 & 76 & 521:1094\footnote{Da sich an diesem Profil nichts ge�ndert hatte, wurde der Testlauf nicht noch einmal durchgef�hrt; die Werte stammen aus dem 1. Lauf}\\
			Aggressive & 20 & 4 & 76 & 545:1040\\
			Defensive & 13 & 4 & 83 & 385:1186\\
			Expansive & 19 & 12 & 69 & 578:1010
		\end{tabular}\par
		\vspace{-0.75\skip\footins}
   \renewcommand{\footnoterule}{}
  \end{minipage}
	\caption{Bilanz der 2. Testl�ufe gegen xathis}
	\label{tab:testAgainstXathis2}
\end{table}

Aus Tabelle \ref{tab:testAgainstXathis2} sieht man, dass die neuen Profile gegen xathis eher schlechter abschnitten, aber die Unterschiede zum 1. Testlauf sind nur gering.

\begin{table}[H]
	\centering
		\begin{tabular}{ l | r  r  r  r }
			\textbf{Profil} & \textbf{Siege} & \textbf{Unentschieden} & \textbf{Niederlagen} & \textbf{Total Punkte} \\
			\hline
			Default & 41 & 9 & 50 & 428:788\\
			Aggressive & 12 & 7 & 81 & 276:940\\
			Defensive & 22 & 6 & 72 & 256:960\\
			Expansive & 11 & 10 & 79 & 256:960
		\end{tabular}
	\caption{Bilanz der 2. Testl�ufe gegeneinander}
	\label{tab:test4Profile2}
\end{table}

Wie Tabelle \ref{tab:test4Profile2} zeigt, waren die Unterschiede zwischen den neuen Profilen im Kampf gegeneinander schon deutlich sichtbarer. Die extremeren Profile weisen alle eine deutlich schlechtere Bilanz auf, wobei die Defensive Konfiguration noch am besten f�hrt. Gegen diese Profile ist unsere ausgewogene Standardkonfiguration aber bereits ein deutlicher Sieger.%L DONE
\section{Online Tests}
\label{sec:testCenter.Online}
Anfangs Dezember stellten wir fest, dass ein User aus dem alten AI-Challenge Forum mit dem Namen ''smiley1983`` einen TCP-Server aufgesetzt hatte, �ber welchen man Bots gegeneinander antreten lassen konnte. Nat�rlich probierten wir das aus und f�hrten auch um die 50 Spiele �ber diesen Server aus. F�r aussagekr�ftige Tests war die Teilnehmerzahl auf diesem Server aber zu gering, weshalb wir uns dagegen entschieden, diese neue Option in unser Testkonzept aufzunehmen.%L DONE

\chapter{Spielanleitung}
\label{chap:spielanleitung}

\section{Systemvoraussetzungen}
\label{chap:spielanleitung.voraussetzungen}

Um ein Spiel ausf�hren zu k�nnen, muss auf dem Computer folgende Software installiert sein:
\begin{enumerate}
\item Java SE JDK Version 6 \footnote{\url{http://www.oracle.com/technetwork/java/javasebusiness/downloads/java-archive-downloads-javase6-419409.html}}
\item Python \footnote{\url{http://www.python.org/download/}}
\item ANT \footnote{\url{http://ant.apache.org/bindownload.cgi}}
\end{enumerate}

Falls das Spiel mit anderen als den vorkonfigurierten Gegnern gestartet werden soll, muss evtl. noch andere Software installiert werden, je nachdem in welcher Programmiersprache die entsprechenden Bots geschrieben sind.

\section{Ausf�hren eines Spiels}
\label{chap:spielanleitung.Ausfuehrung}
  
Im Ordner \texttt{Code} befinden sich unsere Eclipse-Projekte mit dem gesamten Source-Code unserer Implementation, und der offiziellen Spiel-Engine. Das Hauptprojekt befindet sich im Unterordner \texttt{Ants}. Zum einfachen Ausf\"uhren eines Spiels haben wir dort ein ANT-Buildfile (build.xml) erstellt. Dieses definiert verschiedene Targets, mit denen ein Spiel mit jeweils unterschiedlichen Parametern gestartet werden kann.
Falls ANT korrekt installiert ist, k�nnen diese Targets aus dem \texttt{Ants}-Verzeichnis heraus einfach mit ''\texttt{ant <targetName>}`` aufgerufen werden. 
\begin{enumerate}
\item
Das Target \texttt{testBot} ist lediglich zum einfachen Testen eines Bots sinnvoll und entspricht dem Spiel, das verwendet wird, um Bots, die auf der Website hochgeladen werden, zu testen.
\item
Das Target \texttt{runTutorial} f\"uhrt ein Spiel mit den Parametern aus, die im Tutorial auf der Website zur Erkl\"arung der Spielmechanik verwendet werden.
\item
Die Targets \texttt{mazeAgainst*} f\"uhren jeweils ein Spiel auf einer komplexeren und gr\"osseren, labyrinthartigen Karte aus, und zwar gegen den jeweils bezeichneten Gegner.
\item
Das Target \texttt{maze4Players} f�hrt ein Spiel gegen drei andere Bots auf einer noch etwas gr�sseren Karte durch.
\item
Die Targets \texttt{mazeProfiles} und \texttt{mazeProfiles4Players} sind dazu da, mehrere Kopien unseres Bots mit verschiedenen Profilen gegeneinander antreten zu lassen. Wie im Kapitel \ref{sec:module.Logging.Profile} erw�hnt, muss dazu erst in der Klasse LiveInfo das LiveInfo-Logging deaktiviert werden: 
\lstset{language=Java, tabsize=4}
\begin{lstlisting}
private static boolean enabled = false;
\end{lstlisting}
\item 
Das Target \texttt{testOnline} l�sst den Bot �ber einen privaten TCP-Server gegen andere Bots antreten, unter �hnlichen Bedingungen wie beim eigentlichen Wettbewerb, nur leider mit deutlich weniger Teilnehmern. (s. Kapitel \ref{sec:testCenter.Online})
\item
Die Targets \texttt{repeat*} dienten der Durchf�hrung der Profil-Testl�ufe und sind ansonsten wenig interessant.
\item
Die Targets \texttt{debug*} dienten dem Starten unserer Testbots, die wir w�hrend der Entwicklung einsetzten (s. Kapitel \ref{sec:testCenter.Testbots}).
\end{enumerate}

Bei den meisten dieser Targets erscheint beim Starten eine Abfrage, welches Profil man gerne starten w�rde. Hier stehen die in Kapitel \ref{sec:module.Profile.DefinierteProfile} beschriebenen Profile zur Auswahl.

Im Unterordner \texttt{tools} befindet sich die in Python implementierte Spiel-Engine. Unter \texttt{tools/maps} liegen noch weitere vordefinierte Umgebungen, und unter \texttt{tools/mapgen} liegen verschiedene Map-Generatoren, die zur Erzeugung beliebiger weiterer Karten verwendet werden k\"onnen. 

Im Unterordner \texttt{tools/sample\_bots} befinden sich einige einfache Beispiel-Bots, gegen die man spielen kann. 

Im Unterordner \texttt{tools/bots} haben wir Kopien der Bots von einigen Teilnehmern abgelegt, gegen die wir im Zug der Entwicklung immer wieder getestet haben. %L DONE
%todo:
%Glossar
%Bibliographie
%Influence Map or InfluenceMap --> Glossar
%Entscheidungen dokumentieren (warum dieser Algorithmus, warum nicht dieses Design Pattern, ...)
%verlinkung
%einheitliches Tabellendesign
%einheitliche Listings mit caption
%literaturverzeichnisngs 
%deklarieren list java  code pseudo code)
%Bibliographie: papers influence map , hpa* und pathsmoothing

% Attachment:
%---------------------------------------------------------------------------
%\appendix
%\settocdepth{section}
%\chapter{Spielanleitung}
\label{chap:spielanleitung}

\section{Systemvoraussetzungen}
\label{chap:spielanleitung.voraussetzungen}

Um ein Spiel ausf�hren zu k�nnen, muss auf dem Computer folgende Software installiert sein:
\begin{enumerate}
\item Java SE JDK Version 6 \footnote{\url{http://www.oracle.com/technetwork/java/javasebusiness/downloads/java-archive-downloads-javase6-419409.html}}
\item Python \footnote{\url{http://www.python.org/download/}}
\item ANT \footnote{\url{http://ant.apache.org/bindownload.cgi}}
\end{enumerate}

Falls das Spiel mit anderen als den vorkonfigurierten Gegnern gestartet werden soll, muss evtl. noch andere Software installiert werden, je nachdem in welcher Programmiersprache die entsprechenden Bots geschrieben sind.

\section{Ausf�hren eines Spiels}
\label{chap:spielanleitung.Ausfuehrung}
  
Im Ordner \texttt{Code} befinden sich unsere Eclipse-Projekte mit dem gesamten Source-Code unserer Implementation, und der offiziellen Spiel-Engine. Das Hauptprojekt befindet sich im Unterordner \texttt{Ants}. Zum einfachen Ausf\"uhren eines Spiels haben wir dort ein ANT-Buildfile (build.xml) erstellt. Dieses definiert verschiedene Targets, mit denen ein Spiel mit jeweils unterschiedlichen Parametern gestartet werden kann.
Falls ANT korrekt installiert ist, k�nnen diese Targets aus dem \texttt{Ants}-Verzeichnis heraus einfach mit ''\texttt{ant <targetName>}`` aufgerufen werden. 
\begin{enumerate}
\item
Das Target \texttt{testBot} ist lediglich zum einfachen Testen eines Bots sinnvoll und entspricht dem Spiel, das verwendet wird, um Bots, die auf der Website hochgeladen werden, zu testen.
\item
Das Target \texttt{runTutorial} f\"uhrt ein Spiel mit den Parametern aus, die im Tutorial auf der Website zur Erkl\"arung der Spielmechanik verwendet werden.
\item
Die Targets \texttt{mazeAgainst*} f\"uhren jeweils ein Spiel auf einer komplexeren und gr\"osseren, labyrinthartigen Karte aus, und zwar gegen den jeweils bezeichneten Gegner.
\item
Das Target \texttt{maze4Players} f�hrt ein Spiel gegen drei andere Bots auf einer noch etwas gr�sseren Karte durch.
\item
Die Targets \texttt{mazeProfiles} und \texttt{mazeProfiles4Players} sind dazu da, mehrere Kopien unseres Bots mit verschiedenen Profilen gegeneinander antreten zu lassen. Wie im Kapitel \ref{sec:module.Logging.Profile} erw�hnt, muss dazu erst in der Klasse LiveInfo das LiveInfo-Logging deaktiviert werden: 
\lstset{language=Java, tabsize=4}
\begin{lstlisting}
private static boolean enabled = false;
\end{lstlisting}
\item 
Das Target \texttt{testOnline} l�sst den Bot �ber einen privaten TCP-Server gegen andere Bots antreten, unter �hnlichen Bedingungen wie beim eigentlichen Wettbewerb, nur leider mit deutlich weniger Teilnehmern. (s. Kapitel \ref{sec:testCenter.Online})
\item
Die Targets \texttt{repeat*} dienten der Durchf�hrung der Profil-Testl�ufe und sind ansonsten wenig interessant.
\item
Die Targets \texttt{debug*} dienten dem Starten unserer Testbots, die wir w�hrend der Entwicklung einsetzten (s. Kapitel \ref{sec:testCenter.Testbots}).
\end{enumerate}

Bei den meisten dieser Targets erscheint beim Starten eine Abfrage, welches Profil man gerne starten w�rde. Hier stehen die in Kapitel \ref{sec:module.Profile.DefinierteProfile} beschriebenen Profile zur Auswahl.

Im Unterordner \texttt{tools} befindet sich die in Python implementierte Spiel-Engine. Unter \texttt{tools/maps} liegen noch weitere vordefinierte Umgebungen, und unter \texttt{tools/mapgen} liegen verschiedene Map-Generatoren, die zur Erzeugung beliebiger weiterer Karten verwendet werden k\"onnen. 

Im Unterordner \texttt{tools/sample\_bots} befinden sich einige einfache Beispiel-Bots, gegen die man spielen kann. 

Im Unterordner \texttt{tools/bots} haben wir Kopien der Bots von einigen Teilnehmern abgelegt, gegen die wir im Zug der Entwicklung immer wieder getestet haben. 
%\include{anhang/beispielanhangB}
%---------------------------------------------------------------------------

% Glossary
%---------------------------------------------------------------------------
%\cleardoublepage
%\phantomsection 
%\addcontentsline{toc}{chapter}{Glossar}
%\renewcommand{\glossaryname}{Glossar}
%\printglossary
%---------------------------------------------------------------------------

 Bibliography
---------------------------------------------------------------------------
\cleardoublepage
\phantomsection 
\addcontentsline{toc}{chapter}{Literaturverzeichnis}
\bibliographystyle{IEEEtranS}
\bibliography{datenbanken/bibliography}{}
---------------------------------------------------------------------------

% Index
%---------------------------------------------------------------------------
%\cleardoublepage
%\phantomsection 
%\addcontentsline{toc}{chapter}{Stichwortverzeichnis}
%\renewcommand{\indexname}{Stichwortverzeichnis}
%\printindex
%---------------------------------------------------------------------------

%---------------------------------------------------------------------------
\end{document}


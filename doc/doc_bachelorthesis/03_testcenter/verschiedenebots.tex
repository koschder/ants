\section{Testbots}
\label{sec:testCenter.Testbots}

Eine weitere Method war Testbots zu erstellen. Zum Beispiel haben wir den DefendHillBot erstellt der nur verteidigt nicht aber angreift. Wir nahmen eine kleine Karte, so dass es schnell zu Angriffen des Gegners kam. So konnten wir unser Verhalten in der Verteidigung nach kurzer Spieldauer genau analysieren und verbessern.\\
\\
Das selbe galt f�r den AttackHillBot. Wir haben wiederum eine kleine Karte genommen auf der viel Futter vorhand war, so dass sich das Ameisenvolk schnell vermehren konnte. Dank der beschr�nkten Karte kam es schnell zu Angriffen, wir konnten unsere Angriffspositionierung testen.\\
\\
Im Kapitel Task wurde beschrieben, welche Task bzw. Missionen nicht erfolgsversprechend waren und wir nicht weiterverfolgt haben. Bevor wir das  aber wussten, haben wir, um die Funktionen zu testen, einen speziellen Bot erstellt. Anhand der Ergebnissen konnten wir herausfinden, dass diese Methoden nicht praktikabel waren und wir die Ideen verworfen haben. So sind im Code noch folgende Bots zu finden:
\begin{itemize}
	\item ConcentrateBot
	\item FlockBot
	\item SwarmBot		
\end{itemize}





\section{Testbots}
\label{sec:testCenter.Testbots}

Eine weitere Testmethode war, Testbots zu erstellen. Zum Beispiel haben wir den DefendHillBot erstellt. Dieser hat die Eigenschaft, nur zu verteidigen, nicht aber anzugreifen. Wir nahmen eine kleine Karte, so dass es schnell zu Angriffen des Gegners kam. Dadurch konnten wir unser Verteidigungvserhalten nach nur kurzer Spieldauer genau analysieren und verbessern.

Das selbe galt f�r den AttackHillBot. Wir haben wiederum eine kleine Karte genommen auf der viel Futter vorhanden ist, so dass sich das Ameisenvolk schnell vermehrt. Dank der beschr�nkten Karte kam es schnell zu Angriffen, und wir konnten so unsere Berechnung der  Angriffsformationen testen.

Im Kapitel Task wurde beschrieben, welche Task bzw. Missionen nicht erfolgversprechend waren und wir nicht weiterverfolgt haben. Bevor wir das aber wussten, haben wir, um die Funktionen zu testen, wiederum einen speziellen Bot erstellt. Anhand der Ergebnisse konnten wir herausfinden, dass diese Methoden nicht praktikabel waren und wir die Ideen verworfen haben. So sind im Code noch folgende Bots zu finden:
\begin{itemize}
	\item ConcentrateBot
	\item FlockBot
	\item SwarmBot		
\end{itemize}
F�r diese Bots sind zwar noch ANT-Targets\footnote{s. Kapitel \ref{chap:spielanleitung.Ausfuehrung}} zum Ausf�hren der Tests vorhanden, da die Bots aber, nachdem sie ihren Zweck erf�llt hatten, nicht mehr weiterentwickelt wurden, lassen sich mit den meisten Bots keine Spiele mehr starten.




\section{Testreport Profile}
\label{sec:testCenter.TestreportProfile}
 
Um die verschiedenen Profile unseres Bots zu testen, f�hrten wir diverse Testl�ufe durch, in denen wir die verschieden konfigurierten Bots jeweils 100 Mal gegen verschiedene Gegner und gegeneinander antreten liessen. F�r diese Testl�ufe wurden die in Tabelle \ref{tab:definierteProfile} aufgef�hrten Profile verwendet.

W�hrend der Entwicklung f�hrten wir die meisten Tests gegen den Bot von Evan Greavette (Username egreavette) durch. Er hatte seinen Bot �ber GitHub ver�ffentlicht: \url{https://github.com/egreavette/Ants-AI}. Daher f�hrten wir auch die ersten Testl�ufe gegen diesen Gegner durch:

\renewcommand{\arraystretch}{1.5}
\begin{table}[H]
	\centering
		\begin{tabular}{ l | r  r  r  r }
			\textbf{Profil} & \textbf{Siege} & \textbf{Unentschieden} & \textbf{Niederlagen} & \textbf{Total Punkte} \\
			\hline
			Default & 62 & 13 & 25 & 957:579\\
			Aggressive & 61 & 18 & 21 & 1005:642\\
			Defensive & 53 & 15 & 32 & 915:726\\
			Expansive & 59 & 16 & 25 & 955:673
		\end{tabular}
	\caption{Bilanz der Testl�ufe gegen egreavette}
	\label{tab:testAgainstEgreavette}
\end{table}

Wie man unschwer erkennen kann, sind gegen diesen Gegner drei der Profile ungef�hr gleichwertig; lediglich der Defensive Bot f�llt etwas ab.

Als n�chstes f�hrten wir einen Testlauf gegen den Sieger des Wettbewerbs durch. Mathis Lichtenberger (Username xathis) hatte seinen Bot ebenfalls �ber GitHub zur Verf�gung gestellt: \url{https://github.com/xathis/AI-Challenge-2011-bot}. 

\begin{table}[H]
	\centering
		\begin{tabular}{ l | r  r  r  r }
			\textbf{Profil} & \textbf{Siege} & \textbf{Unentschieden} & \textbf{Niederlagen} & \textbf{Total Punkte} \\
			\hline
			Default & 17 & 7 & 76 & 521:1094\\
			Aggressive & 23 & 11 & 66 & 593:1010\\
			Defensive & 11 & 8 & 81 & 479:1106\\
			Expansive & 24 & 6 & 70 & 610:988
		\end{tabular}
	\caption{Bilanz der Testl�ufe gegen xathis}
	\label{tab:testAgainstXathis}
\end{table}

Gegen diesen starken Gegner zeigen sich bereits deutlichere Unterschiede zwischen den Profilen. Man sieht, dass die beiden offensiver ausgerichteten Profile (Aggressive und Expansive) sich deutlich besser schlagen, w�hrend das Defensive Profil klar abf�llt. Dies entspricht in etwa unseren Erwartungen. Die St�rken von xathis' Bot liegen eindeutig in der Offensive, w�hrend die Defensive nicht das Prunkst�ck unseres Bots ist. Deshalb k�nnen wir in der Verteidigung gegen xathis fast nur verlieren und haben eine gr�ssere Chance, wenn wir selber in die Offensive gehen.

Als n�chstes liessen wir unsere 4 Profile gegeneinander antreten:

\begin{table}[H]
	\centering
		\begin{tabular}{ l | r  r  r  r }
			\textbf{Profil} & \textbf{Siege} & \textbf{Unentschieden} & \textbf{Niederlagen} & \textbf{Total Punkte} \\
			\hline
			Default & 22 & 7 & 71 & 332:891\\
			Aggressive & 23 & 10 & 67 & 314:909\\
			Defensive & 23 & 8 & 69 & 300:923\\
			Expansive & 16 & 7 & 77 & 277:946
		\end{tabular}
	\caption{Bilanz der Testl�ufe gegeneinander}
	\label{tab:test4Profile}
\end{table}

Hier zeigt sich, dass die Profile im Kampf gegeneinander relativ ausgeglichen sind. Es zeigt sich auch, dass das expansive Verhalten sich gegen mehrere Gegner weniger lohnt, da sich dabei die Ameisen eher auf dem Spielfeld verzetteln und man Gefahr l�uft, isolierte Ameisen zu verlieren.

Alles in allem waren aber diese Standard-Profile noch mehr oder weniger gleichwertig. Wir erstellten daher neue, st�rker vom Standard abweichende Profile und f�hrten noch einmal Testl�ufe gegen xathis und gegeineinander durch.

\renewcommand{\arraystretch}{1.5}
\begin{table}[H]
	\centering
		\begin{tabular}{ l | r  r  r  r }
			\textbf{Parameter} & \textbf{Default} & \textbf{Aggressive 2} & \textbf{Defensive 2} & \textbf{Expansive 2}\\
			\hline
			defaultAllocation.gatherFood & 25 & 15 & 25 & 35\\
			defaultAllocation.explore & 25 & 15 & 25 & 35\\
			defaultAllocation.attackHills & 25 & 60 & 10 & 20\\
			defaultAllocation.defendHills & 25 & 10 & 40 & 10\\
			defendHills.startTurn & 30 & 30 & 20 & 50\\
			defendHills.minControlRadius & 8 & 8 & 20 & 8\\
			explore.forceThresholdPercent & 80 & 80 & 80 & 90\\
			explore.forceGain & 0.25 & 0.1 & 0.25 & 0.4\\
			explore.dominantPositionBoost & 5 & 3 & 5 & 10\\
			attackHills.dominantPositionBoost & 2 & 10 & 2 & 2\\
			attackHills.halfTimeBoost & 20 & 40 & 15 & 20\\
		\end{tabular}
		\caption{Die Profile f�r den 2. Testlauf}
		\label{tab:definierteProfile2}
\end{table}

\begin{table}[H]
	\centering
	%weird hack to enable footnotes in the table
	\begin{minipage}{11cm}
    \centering
		\begin{tabular}{ l | r  r  r  r }
			\textbf{Profil} & \textbf{Siege} & \textbf{Unentschieden} & \textbf{Niederlagen} & \textbf{Total Punkte} \\
			\hline
			Default & 17 & 7 & 76 & 521:1094\footnote{Da sich an diesem Profil nichts ge�ndert hatte, wurde der Testlauf nicht noch einmal durchgef�hrt; die Werte stammen aus dem 1. Lauf}\\
			Aggressive & 20 & 4 & 76 & 545:1040\\
			Defensive & 13 & 4 & 83 & 385:1186\\
			Expansive & 19 & 12 & 69 & 578:1010
		\end{tabular}\par
		\vspace{-0.75\skip\footins}
   \renewcommand{\footnoterule}{}
  \end{minipage}
	\caption{Bilanz der 2. Testl�ufe gegen xathis}
	\label{tab:testAgainstXathis2}
\end{table}

Aus Tabelle \ref{tab:testAgainstXathis2} sieht man, dass die neuen Profile gegen xathis eher schlechter abschnitten, die Unterschiede zum 1. Testlauf sind aber nur gering.

\begin{table}[H]
	\centering
		\begin{tabular}{ l | r  r  r  r }
			\textbf{Profil} & \textbf{Siege} & \textbf{Unentschieden} & \textbf{Niederlagen} & \textbf{Total Punkte} \\
			\hline
			Default & 41 & 9 & 50 & 428:788\\
			Aggressive & 12 & 7 & 81 & 276:940\\
			Defensive & 22 & 6 & 72 & 256:960\\
			Expansive & 11 & 10 & 79 & 256:960
		\end{tabular}
	\caption{Bilanz der 2. Testl�ufe gegeneinander}
	\label{tab:test4Profile2}
\end{table}

Wie Tabelle \ref{tab:test4Profile2} zeigt, waren die Unterschiede zwischen den neuen Profilen im Kampf gegeneinander schon deutlich sichtbarer. Die extremeren Profile weisen alle eine deutlich schlechtere Bilanz auf, wobei die Defensive Konfiguration noch am besten davon kommt. Gegen diese Profile ist unsere ausgewogene Standardkonfiguration aber bereits ein deutlicher Sieger.

\subsection{TestReader}
\label{sec:module.Testreader}
F�r die Auswertung der Testl�ufe haben wir ein kleines Zusatzmodul geschrieben, das aus den Spielberichten der Spielengine die relevanten Informationen auslesen kann. Es besteht lediglich aus einer einzelnen Klasse (\texttt{TestReader} mit einer main() Methode. Es liest aus dem Log-Verzeichnis die *.replay Dateien (Spielberichte im Json-Format) ein, parsed sie mit Hilfe der Simple-Json Bibliothek und erstellt eine CSV-Datei mit den Schlussergebnissen aus den einzelnen Spielen. Ausserdem wird eine zus�tzlich CSV-Datei ausgegeben, die die Ergebnisse aggregiert, und zwar ungef�hr in der Form, wie sie in diesem Kapitel in den Tabellen ersichtlich ist.
\chapter{Rahmenbedingungen}

\section{Spielbeschrieb}
\label{sec:spielbeschrieb}

\subsection{Der Wettbewerb}
\label{sec:spielbeschrieb_wettbewerb}
Die AI Challenge\footnote{\url{http://www.aichallenge.org}} ist ein internationaler Wettbwerb des University of Waterloo Computer Science Club der im Zeitraum Herbst 2011 bis Januar 2012 zum 3. Mal stattgefunden hat. Das Spiel ist ein zugbasiertes Multiplayerspiel in welchem sich Ameisenv�lker gegenseitig bek�mpfen. Ziel einer AI-Challenge ist es, einen Bot zu schreiben, der die gegebenen Aufgaben mit m�glichst intelligenten Algorithmen l�st. Die zu l�senden Aufgaben der Ants AI Challenge sind die Futtersuche, das Explorieren der Karten, das Angreifen von gegnerischen V�lkern und deren Ameisenhaufen sowie dem Sch�tzen des eigenen Ameisenhaufen.

\subsection{Spielregeln}
\label{sec:spielbeschrieb_spielregeln}
Nachfolgend sind die wichtigsten Regeln, die w�hrend dem Spiel ber�cksichtigt werden m�ssen, aufgelistet.
\begin{itemize} 
\item Pro Zug k�nnen alle Ameisen um ein Feld (vertikal oder horizontal) verschoben werden.
\item Pro Zug steht insgesamt eine Rechenzeit von einer Sekunde zur Verf�gung. Es d�rfen keine Threads erstellt werden.
\item Bewegt sich eine Ameise in die 4er Nachbarschaft eines Futterpixel, wird dieses eingesammelt. Beim n�chsten Zug entsteht bei dem Ameisenh�gel eine neu Ameise.
\item Die Landkarte besteht aus passierbaren Landpixel sowie unpassierbaren Wasserstellen.
\item Ein Gegener wird geschlagen, wenn im Kampfradius der eigenen Ameise mehr eigene Ameise stehen als gegnerische Ameisen im Kampfradius der Ameise die angegriffen wird.
\item Ein Gegner ist ausgeschieden wenn alle seine eigenen Ameisenh�gel vom Gegner vernichtet wurden. Pro verlorenem H�gel gib es einen Punkteabzug. Pro feindlichen H�gel, der zerst�rt wird gibt es zwei Bonuspunkte.
\item Steht nach einer definierbaren Zeit (Anzahl Z�ge) kein Sieger fest, wird der Sieger anhand der Punkte ermittelt. 
\end{itemize}
Die ausf�hrlichen Regeln k�nnen auf der Webseite nachgelesen werden: \url{http://aichallenge.org/specification.php}

\subsection{Schnittstelle}
\label{sec:spielbeschrieb_schnittstelle}
Die Spielschnittstelle ist simpel gehalten. Nach jeder Spielrunde erh�lt der Bot das neue Spielfeld mittels String-InputStream, die Spielz�ge gibt der Bot dem Spielcontroller mittels String-OutputStream bekannt. Unser MyBot leitet von der Basis-Klasse Bot\footnote{Die Klasse ist im Code unter ants.bot.Bot.Java auffindbar } ab. Ein Spielzug wird im folgendem Format in den Output-Stream gelegt:
\newline
\newline
o <Zeile> <Spalte> <Richtung>
\newline
\newline
Beispiel:
\begin{verbatim}
o 4 7 W
\end{verbatim}
Die Ameise wird von der Position Zeile 4 und Spalte 7 nach Westen bewegt.
\newline
Der Spielcontroller ist in Python realisiert, der Bot kann aber in allen g�ngigen Programmiersprachen wie Java, Python, C\#, C++ etc. geschrieben werden.

\section{Verwendete Software}
\label{sec:software}

Als Entwicklungsumgebung wird Eclipse verwendet. Die Programmierung erfolgt in Java. Die Bachelorarbeit baut auf dem Codestand des Projekt 2 auf. Dieser Stand basiert auf dem Java-Starter-Paket, welches wir von der AI-Challenge Homepage herunter luden und weiterentwickelten. Das Starter-Paket beinhaltet alle n�tigen Komponenten um ein Spiel durchzuf�hren und anzuschauen.

\section{Verwendete Hardware}
\label{sec:hardware}

Es ist keine spezielle Hardware von N�ten. Das Spiel kann auf einem handels�blichen Computer / Laptop ausgef�hrt werden.

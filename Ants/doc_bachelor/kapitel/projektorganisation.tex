\chapter{Projektorganisation}

\section{Betroffene Personen}
\label{sec:projektorganisation.BetroffenePersonen}

\textbf{Studierende:}\\
 Lukas Kuster \textit{kustl1@bfh.ch} \\
 Stefan K�ser \textit{kases1@bfh.ch}
 
\textbf{Betreuung:}\\
Dr. J�rgen Eckerle \textit{juergen.eckerle@bfh.ch}

\textbf{Experte:}\\
Dr. Frederico Fl�ckiger	\textit{TODO EMAIL}


\section{Projektmeetings}
\label{sec:projektorganisation.Projektmeetings}

\begin{itemize}
\item Es findet ein Treffen mit Betreuer alle 1-2 Wochen statt.
\item Ein Treffen mit dem Experte wird am Anfang und am Ende der Arbeit statt, oder auf Wunsch des Experten.
\end{itemize}

\section{Dokumentation}
\label{sec:projektorganisation.Dokumentation}

Die Dokumentation soll sich am Aufbau und Inhalt des Berichts aus dem Projekt 2 anlehnen.
\begin{itemize}
\item Das Dokument beschr�nkt sich auf das Wesentliche.
\item Entscheidungen und deren Grundlagen sind dokumentiert.
\item Testberichte dokumentieren die durchgef�hrten Modultests. 
\item Klassendiagrammen sollen einen oberfl�chlichen Detailierungsgrad haben, so dass das Wichtigste auf den ersten Blick sichtbar ist.
\end{itemize}

\section{Abgabe}
\label{sec:projektorganisation.Abgabe}
Folgende Lieferobjekte werden am Ende der Arbeit abgegeben.
\begin{itemize}
\item Dokumentation
\item Sourcecode
\end{itemize}

\section{Zeitplan}
\label{sec:projektorganisation.Zeitplan}

TODO

\newcommand{\vspacecm}{0.7cm}

\chapter{Ziele}

Der im Rahmen von Projekt 2 entwickelte Bot soll um Logik f�r taktische und strategische Entscheidungen und koordinierte Bewegung erweitert werden.

\section{Funktionale Anforderungen}
\label{sec:ziele.FunktionaleAnforderungen}

\subsection{Musskriterien}
\label{sec:ziele.FunktionaleAnforderungen.Musskriterien}

Der Bot unterscheidet zwischen diversen Aufgaben:
	\begin{itemize}
	\item Nahrungsbeschaffung
	\item Angriff
	\item Verteidigung
	\item Erkundung
	\end{itemize}

\vspace{\vspacecm}

Der Bot identifiziert zur Erf�llung dieser Aufgaben konkrete Ziele, wie z.B.:
	\begin{itemize}
	\item Gegnerische H�gel angreifen, was bei Erfolg den Score erh�ht und das eigentliche Ziel des Spiels ist.
	\item Isolierte gegnerische Ameisen angreifen.
	\item Schwachstellen in der gegnerischen Verteidigung ausnutzen.
	\item Engp�sse im Terrain sichern bzw. versperren.
	\item Konfliktzonen, d.h. viele Ameisen auf einem engen Raum, erkennen und entsprechend reagieren.
	\end{itemize}
\vspace{\vspacecm}

Die Auswahl von Taktik und Strategie basiert auf der Bewertung der Situation auf dem Spielfeld, z.B. anhand folgender Kriterien:
	\begin{itemize}
	\item Dominante/unterlegene Position
	\item Sicherheit verschiedener Gebiete des Spielfelds (eigener/gegnerischer Einfluss)
	\item Konfliktpotenzial in verschiedenen Gebieten des Spielfelds
	\end{itemize}
		
Anhand der Situationsbeurteilung werden die unterschiedlichen Aufgaben entsprechend gewichtet. Stark gewichtete Aufgaben erhalten mehr Ressourcen (Ameisen) zur Durchf�hrung.


\vspace{\vspacecm}

Die Situationsbeurteilung fliesst auch in die taktische Logik ein, wie folgende Beispiele illustrieren:
	\begin{itemize}
	\item Bei der Pfadsuche wird die Sicherheit der zu durchquerenden Gebiete ber�cksichtigt
	\item In Kampfsituationen kann der Bot die Ameisen in Formationen gliedern, die geeignet sind, eine lokale �berzahl eigener gegen�ber gegnerischen Ameisen zu erzeugen
	\item Beim Aufeinandertreffen mit gegnerischen Ameisen wird entschieden, ob angegriffen, die Stellung gehalten oder gefl�chtet wird.
	\end{itemize}

\subsection{Kannkriterien}
\label{sec:ziele.FunktionaleAnforderungen.Kannkriterien}
Das Verhalten des Bots ist konfigurierbar, so dass zum Beispiel ein �agressiver� Bot gegen einen defensiven Bot antreten kann.

\section{Nicht funktionale Anforderungen}
\label{sec:ziele.NichtFunktionaleAnforderungen}

\subsection{Musskriterien}
\label{sec:ziele.NichtFunktionaleAnforderungen.Musskriterien}
Modularer Aufbau f�r eine gute Testbarkeit der Komponenten.

Wichtige Funktionen wie die Pfadsuche und die Berechnung von Influence Maps sollen in separaten Modulen implementiert werden, damit sie auch von anderen Projekten verwendet werden k�nnten.

Die Codedokumentation ist vollst�ndig und dient der Verst�ndlichkeit.

\subsection{Kannkriterien}
\label{sec:ziele.NichtFunktionaleAnforderungen.Kannkriterien}
F�r die wiederverwendbaren Module wird jeweils ein kleines Tutorial geschrieben, wie die Module verwendbar sind.


\section{Abgrenzungskriterien}

Da der Wettbewerb Ants AI Challenge bereits beendet ist, ist es uns nicht m�glich dem Wettbewerb teilzunehmen.
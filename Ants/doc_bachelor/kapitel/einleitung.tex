\chapter{Einleitung}
\label{chap:einleitung}
Im Rahmen des Moduls ''Projekt 2'' (7302) haben wir uns mit der Implementierung eines Bots f�r den Online-Wettbewerb AI-Challenge (Ants) besch�ftigt. Die AI-Challenge ist ein Wettbewerb, der im Herbst 2011 zum 3. Mal stattfand und jedes Jahr mit einem anderen Spiel durchgef�hrt wird. Ziel ist es jeweils, einen Bot zu programmieren, der durch geschickten Einsatz von KI-Technologien das Spiel m�glichst erfolgreich bestreiten kann. In dieser Durchf�hrung ging es darum, ein Ameisenvolk durch Sammeln von Ressourcen und Erobern von gegnerischen H�geln zum Sieg �ber die gegnerischen Ameisen zu f�hren.

Wir hatten uns zum Ziel gesetzt, einen Bot zu implementieren, der m�glichst alle Bereiche des Spiels beherrscht, also Nahrung sammeln, die Gegend entdecken, H�gel erobern und gegen feindliche Ameisen k�mpfen. Im Gegenzug legten wir kein besonderes Gewicht darauf, dass der Bot eines dieser Verhalten besonders gut beherrschen muss. Das prim�re Ziel war es, Erfahrungen zu sammeln im Hinblick auf die Bachelor-Arbeit.

Den gr�ssten Aufwand bei der Implementierung steckten wir in die Pfadsuche, da diese eine Voraussetzung f�r nahezu alle Teilaufgaben des Bots ist. Nachdem wir mit dem bekannten A*-Algorithmus zwar kleine Erfolge erzielten, aber auch schnell Performance-Probleme bekamen, entschlossen wir uns, auf Basis eines Clustering des Spielfeldes den HPA*-Algorithmus umzusetzen, was zu erheblichen Performance-Verbesserungen f�hrte.

Ein weiterer Punkt, auf den wir viel Wert legten, war die Programmstruktur. Unser Bot ist objektorientiert aufgebaut. Die zentrale Einheit sind die verschiedenen Tasks, die jeweils f�r eine Aufgabe der Ameisen zust�ndig sind. 



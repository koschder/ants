\chapter{Spielanleitung}
\label{chap:spielanleitung}

Dieser Anhang beschreibt kurz, wie ein Spiel mit unserem Bot ausgef\"uhrt werden kann.

Das File \texttt{Ants.zip} enth\"alt das Eclipse-Projekt mit dem gesamten Source-Code unserer Implementation, und der offiziellen Spiel-Engine. Zum einfachen Ausf\"uhren eines Spiels haben wir ein ANT-Buildfile (build.xml) erstellt. Dieses definiert 3 Targets, mit denen ein Spiel mit jeweils unterschiedlichen Parametern gestartet werden kann.
\begin{enumerate}
\item{Das Target \texttt{testBot} ist lediglich zum einfachen Testen eines Bots sinnvoll und entspricht dem Spiel, das verwendet wird, um Bots, die auf der Website hochgeladen werden, zu testen.}

\item{Das Target \texttt{runTutorial} f\"uhrt ein Spiel mit den Parametern aus, die im Tutorial auf der Website zur Erkl\"arung der Spielmechanik verwendet werden.}
\item{
Das Target \texttt{maze} f\"uhrt ein Spiel auf einer komplexeren und gr\"osseren, labyrinthartigen Karte aus und ist das interessanteste von den 3.}

\end{enumerate}

Im Unterordner \texttt{tools} befindet sich die in Python implementierte Spiel-Engine. Unter \texttt{tools/maps} liegen noch weitere vordefinierte Umgebungen, und unter \texttt{tools/mapgen} liegen verschieden Map-Generatoren, die zur Erzeugung beliebiger weiterer Karten verwendet werden k\"onnen. 

Im Unterordner \texttt{tools/sample\_bots} befinden sich einige einfache Beispiel-Bots, gegen die man spielen kann. Viele der Teilnehmer haben zudem ihren Quellcode auf dem Internet publiziert, an m\"oglichen Gegner besteht also auch kein Mangel.
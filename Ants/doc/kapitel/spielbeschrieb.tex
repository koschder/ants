\chapter{Spielbeschrieb}
\label{chap:spielbeschrieb}

\section{Der Wettbewerb}
\label{sec:spielbeschrieb_wettbewerb}
Die AI Challenge\footnote{http://www.aichallenge.org} ist ein internationaler Wettbwerb des University of Waterloo Computer Science Club der im Zeitraum Herbst 2011 bis Januar 2012 stattgefunden hat. Das Spiel ist ein zugbasiertes Multiplayerspiel in welchem Sich Ameisenv�lker gegenseitig bek�mpfen. Ziel einer AI-Challenge ist es, einen m�glichst Bot zu schreiben, der die gegebenen Aufgaben mit m�glichst intelligent Algrithmen l�st. Die zu l�senden Aufgaben der Ants AI Challenge sind die Futtersuche, das Explorieren der Karten, das Angreifen von gegnerischen V�lkern und deren Ameisenhaufen sowie dem Sch�tzen des eigenen Ameisenhaufen.

\section{Spielregeln}
\label{sec:spielbeschrieb_spielregeln}
Nachfolgend sind die wichtigsten Regeln, die w�hrend dem Spiel ber�cksichtigt, werden m�ssen aufgelistet.
\begin{itemize} 
\item Pro Zug k�nnen alle Ameisen um ein Feld (vertikal oder horizontal) verschoben werden.
\item Pro Zug steht insgesamt eine Rechenzeit von einer Sekunde zur Verf�gung.
\item Bewegt sich eine Ameise in die 4er Nachbarschaft eines Futterpixel, wird dieses "gefressen" und beim n�chsten Zug entsteht beim Ameisenh�gel eine neu Ameise
\item Die Landkarte besteht aus passierbaren Landpixel sowie unpassierbaren Wasserstellen
\item Ein Gegener wird geschlagen, wenn im Kampfradius der eigenen Ameise mehr eigene Ameise stehen als gegnerische Ameisen im Kampfradius der Ameise die angegriffen wird.
\item Ein Gegner ist ausgeschieden wenn alle seine eigenen Ameisenh�gel vom Gegner vernichtet wurden. Pro verlorener H�gel wir minus ein Punkt berechnet, pro zerst�rter H�gel plus 2 Punkte.
\item Steht nach einer definierbaren Zeit (Anzahl Z�ge) kein Sieger fest, werden die Punkte gez�hlt. 
\end{itemize}
Die ausf�hrlichen Regeln k�nnen auf der Webseite nachgelesen werden: http://aichallenge.org/specification.php

\section{Schnittstelle}
\label{sec:spielbeschrieb_schnittstelle}
Die Spielschnittstelle ist simpel gehalten. Nach jeder Spielrunde erh�lt der Bot das neue Spielfeld mittels String-InputStream, die Spielz�ge gibt der Bot dem Spielcontroller mittels String-OutputStream bekannt. Unser MyBot leitet vom Interface Bot\footnote{das Interface ist im Code unter ants.bot.Bot.Java auffindbar } ab. Ein Spielzug wird im folgendem Format in den Output-Stream gelegt:
\newline
\newline
o <Zeile> <Spalte> <Richtung>
\newline
\newline
Beispiel:
\begin{verbatim}
o 4 7 W
\end{verbatim}
Die Ameise wird von der Positon Zeile 4 und Spalte 7 nach Westen bewegt.
\newline
Der Spielcontroller ist in Python realisiert, der Bot kann aber in allen g�nglichen Programmiersprachen wie Java, Python, C#, C++ etc. geschrieben werden.
